\section{Shuffle Test}
\label{app-shuf}
Appendix B contains the geometry cross sections, fission rate/thermal flux meshes, and image difference results from the shuffling tests (see \autoref{meth-sens}, Table \ref{table:shuffle}), which were omitted from the main report for brevity.

Comparing the image difference results of Appendix B, the shuffling test, to Appendix A, the symmetry test, shows that the shuffling tests have a weaker effect on the fission rate and thermal flux than the symmetry tests.  The small differences in this particular test would most likely indicate that the core is generally well-mixed, i.e., that each bin in the vertical direction, along the z axis, has each of the 7 fuel compositions represented equally.


\begin{figure}[H]
\centering

\begin{subfigure}{0.45\textwidth}
  \includegraphics[width=0.95\linewidth]{figures/1234560/1234560-r}
  \caption{Radial Cross Section at y=0}
  \label{fig:1234560-r}
\end{subfigure}%
%
\begin{subfigure}{0.45\textwidth}
  \includegraphics[width=0.95\linewidth]{figures/1234560/1234560-rm}
  \caption{Radial Mesh}
  \label{fig:1234560-rm}
\end{subfigure}

\begin{subfigure}{0.45\textwidth}
  \includegraphics[width=0.95\linewidth]{figures/1234560/1234560-v}
  \caption{Axial Cross Section at z=0 }
  \label{fig:1234560-v}
\end{subfigure}
%
\begin{subfigure}{0.45\textwidth}
  \includegraphics[width=0.95\linewidth]{figures/1234560/1234560-vm}
  \caption{Axial Mesh}
  \label{fig:1234560-vm}
\end{subfigure}
%
\caption{Shuffle Analysis: Run 1}
\label{fig:0-60}
\end{figure}
\begin{figure}[H]
\centering
\includegraphics[width=0.6\linewidth]{figures/shuffle/diff-1234560}
\caption{An Image Generated by Subtracting \ref{fig:1234560-rm} from \ref{fig:controlb}.}
\label{fig:diff-1234560}
\end{figure}

Figure \ref{fig:1234560} provides the thermal flux and fission rate meshes and geometric cross sections axially and radially.  Figure \ref{fig:diff-1234560} is the result of the image difference between the full core control mesh and Figure \ref{fig:1234560-rm}.

\begin{figure}[H]
\centering

\begin{subfigure}{0.45\textwidth}
  \includegraphics[width=0.95\linewidth]{figures/2345601/2345601-r}
  \caption{Radial Cross Section at y=0}
  \label{fig:2345601-r}
\end{subfigure}%
%
\begin{subfigure}{0.45\textwidth}
  \includegraphics[width=0.95\linewidth]{figures/2345601/2345601-rm}
  \caption{Radial Mesh}
  \label{fig:2345601-rm}
\end{subfigure}

\begin{subfigure}{0.45\textwidth}
  \includegraphics[width=0.95\linewidth]{figures/2345601/2345601-v}
  \caption{Axial Cross Section at z=0 }
  \label{fig:2345601-v}
\end{subfigure}
%
\begin{subfigure}{0.45\textwidth}
  \includegraphics[width=0.95\linewidth]{figures/2345601/2345601-vm}
  \caption{Axial Mesh}
  \label{fig:2345601-vm}
\end{subfigure}
%
\caption{Shuffle Analysis: Run 2}
\label{fig:0-60}
\end{figure}
\begin{figure}[H]
\centering
\includegraphics[width=0.6\linewidth]{figures/shuffle/diff-2345601}
\caption{An Image Generated by Subtracting Figure \ref{fig:2345601-rm} from Figure \ref{fig:controlb}.}
\label{fig:diff-2345601}
\end{figure}

Figure \ref{fig:2345601} provides the thermal flux and fission rate meshes and geometric cross sections axially and radially.  Figure \ref{fig:diff-2345601} is the result of the image difference between the full core control mesh and Figure \ref{fig:2345601-rm}.

\begin{figure}[H]
\centering

\begin{subfigure}{0.45\textwidth}
  \includegraphics[width=0.95\linewidth]{figures/3456012/3456012-r}
  \caption{Radial Cross Section at y=0}
  \label{fig:3456012-r}
\end{subfigure}%
%
\begin{subfigure}{0.45\textwidth}
  \includegraphics[width=0.95\linewidth]{figures/3456012/3456012-rm}
  \caption{Radial Mesh}
  \label{fig:3456012-rm}
\end{subfigure}

\begin{subfigure}{0.45\textwidth}
  \includegraphics[width=0.95\linewidth]{figures/3456012/3456012-v}
  \caption{Axial Cross Section at z=0 }
  \label{fig:3456012-v}
\end{subfigure}
%
\begin{subfigure}{0.45\textwidth}
  \includegraphics[width=0.95\linewidth]{figures/3456012/3456012-vm}
  \caption{Axial Mesh}
  \label{fig:3456012-vm}
\end{subfigure}
%
\caption{Shuffle Analysis: Run 3}
\label{fig:0-60}
\end{figure}
\begin{figure}[H]
\centering
\includegraphics[width=0.6\linewidth]{figures/shuffle/diff-3456012}
\caption{An Image Generated by Subtracting Figure \ref{fig:3456012-rm} from Figure \ref{fig:controlb}.}
\label{fig:diff-3456012}
\end{figure}

Figure \ref{fig:3456012} provides the thermal flux and fission rate meshes and geometric cross sections axially and radially.  Figure \ref{fig:diff-3456012} is the result of the image difference between the full core control mesh and Figure \ref{fig:3456012-rm}.

\begin{figure}[H]
\centering

\begin{subfigure}{0.45\textwidth}
  \includegraphics[width=0.95\linewidth]{figures/4560123/4560123-r}
  \caption{Radial Cross Section at y=0}
  \label{fig:4560123-r}
\end{subfigure}%
%
\begin{subfigure}{0.45\textwidth}
  \includegraphics[width=0.95\linewidth]{figures/4560123/4560123-rm}
  \caption{Radial Mesh}
  \label{fig:4560123-rm}
\end{subfigure}

\begin{subfigure}{0.45\textwidth}
  \includegraphics[width=0.95\linewidth]{figures/4560123/4560123-v}
  \caption{Axial Cross Section at z=0 }
  \label{fig:4560123-v}
\end{subfigure}
%
\begin{subfigure}{0.45\textwidth}
  \includegraphics[width=0.95\linewidth]{figures/4560123/4560123-vm}
  \caption{Axial Mesh}
  \label{fig:4560123-vm}
\end{subfigure}
%
\caption{Shuffle Analysis: Run 4}
\label{fig:4560123}
\end{figure}
\begin{figure}[H]
\centering
\includegraphics[width=0.6\linewidth]{figures/shuffle/diff-4560123}
\caption{An Image Generated by Subtracting \ref{fig:4560123-rm} from \ref{fig:controlb}.}
\label{fig:diff-4560123}
\end{figure}

Figure \ref{fig:4560123} provides the thermal flux and fission rate meshes and geometric cross sections axially and radially.  Figure \ref{fig:diff-4560123} is the result of the image difference between the full core control mesh and Figure \ref{fig:4560123-rm}.

\begin{figure}[H]
\centering

\begin{subfigure}{0.45\textwidth}
  \includegraphics[width=0.95\linewidth]{figures/5601234/5601234-r}
  \caption{Radial Cross Section at y=0}
  \label{fig:5601234-r}
\end{subfigure}%
%
\begin{subfigure}{0.45\textwidth}
  \includegraphics[width=0.95\linewidth]{figures/5601234/5601234-rm}
  \caption{Radial Mesh}
  \label{fig:5601234-rm}
\end{subfigure}

\begin{subfigure}{0.45\textwidth}
  \includegraphics[width=0.95\linewidth]{figures/5601234/5601234-v}
  \caption{Axial Cross Section at z=0 }
  \label{fig:5601234-v}
\end{subfigure}
%
\begin{subfigure}{0.45\textwidth}
  \includegraphics[width=0.95\linewidth]{figures/5601234/5601234-vm}
  \caption{Axial Mesh}
  \label{fig:5601234-vm}
\end{subfigure}
%
\caption{Shuffle Analysis: Run 5}
\label{fig:5601234}
\end{figure}
\begin{figure}[H]
\centering
\includegraphics[width=0.6\linewidth]{figures/shuffle/diff-5601234}
\caption{An Image Generated by Subtracting Figure \ref{fig:5601234-rm} from Figure \ref{fig:controlb}.}
\label{fig:diff-5601234}
\end{figure}

Figure \ref{fig:5601234} provides the thermal flux and fission rate meshes and geometric cross sections axially and radially.  Figure \ref{fig:diff-5601234} is the result of the image difference between the full core control mesh and Figure \ref{fig:5601234-rm}.

\begin{figure}[H]
\centering

\begin{subfigure}{0.45\textwidth}
  \includegraphics[width=0.95\linewidth]{figures/6012345/6012345-r}
  \caption{Radial Cross Section at y=0}
  \label{fig:6012345-r}
\end{subfigure}%
%
\begin{subfigure}{0.45\textwidth}
  \includegraphics[width=0.95\linewidth]{figures/6012345/6012345-rm}
  \caption{Radial Mesh}
  \label{fig:6012345-rm}
\end{subfigure}

\begin{subfigure}{0.45\textwidth}
  \includegraphics[width=0.95\linewidth]{figures/6012345/6012345-v}
  \caption{Axial Cross Section at z=0 }
  \label{fig:6012345-v}
\end{subfigure}
%
\begin{subfigure}{0.45\textwidth}
  \includegraphics[width=0.95\linewidth]{figures/6012345/6012345-vm}
  \caption{Axial Mesh}
  \label{fig:6012345-vm}
\end{subfigure}
%
\caption{Shuffle Analysis: Run 6}
\label{fig:0-60}
\end{figure}
\begin{figure}[H]
\centering
\includegraphics[width=0.6\linewidth]{figures/shuffle/diff-6012345}
\caption{An Image Generated by Subtracting Figure \ref{fig:6012345-rm} from Figure \ref{fig:controlb}.}
\label{fig:diff-6012345}
\end{figure}

Figure \ref{fig:6012345} provides the thermal flux and fission rate meshes and geometric cross sections axially and radially.  Figure \ref{fig:diff-6012345} is the result of the image difference between the full core control mesh and Figure \ref{fig:6012345-rm}.

There are a few hotspots where small regions show a brighter patch of green.  The best example of this is in the fourth quadrant of Figure \ref{fig:diff-6012345}, near the $270^{\circ}$ line.  Hotspots such as this could be caused by poor mixing, which would make some pebbles occur in higher concentrations (and then cause a larger difference in the shuffle test, when the same region is now dominated by a different burnup) or by the shuffling putting a different pebble burnup in a region of lower or higher flux.  To investigate the source of this, the pebbles in a ~10 cm square column (10 cm in x and y, the height of the reactor in z) around the hotspot were found.  Table \ref{table:10cmpebb} shows the count of each pebble burnup.


\begin{table}[H]
\centering
\caption{Representation of Pebbles by Number of Passes in a 10 cm Square Rectangular Prism Surrounding an Image Difference Hotspot at Approximately x = 11 cm, y = -56 cm}
 \begin{tabularx}{0.35\textwidth}{c  c}
 	\hline
 	Pebble Pass & Number of Pebbles \\
 	\hline
 	Fresh & 20 \\
 	First Pass & 12 \\
 	Second Pass & 13 \\
 	Third Pass & 12 \\
 	Fourth Pass & 10 \\
 	Fifth Pass & 14 \\
 	Sixth Pass & 11 \\
 	\hline
 \end{tabularx}
\label{table:10cmpebb}
\end{table}

When this region was narrowed further, to a ~5 cm region, the poor mixing was more dramatic, as seen in Table \ref{table:5cmpebb}.


\begin{table}[H]
\centering
\caption{Representation of Pebbles by Number of Passes in a 5 cm Square Rectangular Prism Surrounding an Image Difference Hotspot at Approximately x = 11 cm, y = -56 cm}
 \begin{tabularx}{0.35\textwidth}{c  c}
 	\hline
 	Pebble Pass & Number of Pebbles \\
 	\hline
 	Fresh & 10\\
 	First Pass & 3 \\
 	Second Pass & 4 \\
 	Third Pass & 1 \\
 	Fourth Pass & 2 \\
 	Fifth Pass & 2 \\
 	Sixth Pass & 2 \\
 	\hline

 \end{tabularx}
\label{table:5cmpebb}
\end{table}

In the 10 cm square column, fresh pebbles originally make up 21.7\% of all pebbles in the region.  In the 5cm square column that is tighter around the hotspot, fresh pebbles make up 41.7\% of all pebbles.  The hotspot pointed out in Figure \ref{fig:diff-6012345} exists in some degree in Figures \ref{fig:diff-1234560}, \ref{fig:diff-2345601}, \ref{fig:diff-3456012}, \ref{fig:diff-4560123}, and \ref{fig:diff-5601234} --- it is simply brightest in Figure \ref{fig:diff-6012345} because the shuffle scheme corresponding to this Figure \ref{fig:diff-6012345} replaces fresh pebbles with 6-pass pebbles, which have the greatest disparity in burnup.