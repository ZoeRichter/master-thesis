

\section{Motivation}

The effects of global warming are becoming increasingly severe. (***include NASA facts?***).  In the interest of reducing the global carbon footprint, a desire for carbon-free, sustainable energy is growing. With this interest comes a bevy of new research in the next generation of nuclear reactors.

One such class of reactors are the high temperature gas-cooled reactor, or HTGR.  While HTGRs can have a variety of fuel forms, of particular interest are pebble bed reactors.  A pebble-type fuel generally consists of a sphere of graphite, approximately the size of a billard ball, embedded with TRISO particles.  The fuel kernels in these TRISO particles are surrounded in multiple layers of carbon and silicon carbide, and, along with the graphite that creates the sphere proper, form a durable, compact fuel form.  In addition, the pebbles are able to be refueled online, reducing the need for planned shutdowns.

The next generation of nuclear reactors also include designs significantly smaller than the conventional Light Water Reactor(LWR) seen in the USA today.  So-called Small Modular Reactors, or SMRs, these reactors are small enough to be shipped, reactor pressure vessel and all, in a standard shipping truck or train.  The pressure vessels can also be produced in a factory of standard size (**** I dislike this wording. but can't think of another way to put it****).  SMRs can be deployed in a variety of new settings, such as isolated towns or work sites, or many can be stationed together in one plant to fill the role of a single larger reactor.

This work used a pebble-bed HTG-SMR as a starting point, and modeled a fairly generic 200MWth reactor based on existing designs - named Sangamon200. Then it scaled down to a target size - a 20MWth pebble bed HTGR.  "Microreactors" such as these are generally 70 MWth or less, and can be deployed in areas where only a small amount of power is needed, used for research and testing, or be used to supply heat for other industrial processes, such as producing hydrogen (*****roberto pres links find*****).

The 20MWth model, which will hereafter be referred to as Sangamon20, is of a highly simplified design, which can be used in future testing and analysis.

*******why modular: reutler and lohnert**********

\section{Objectives}

