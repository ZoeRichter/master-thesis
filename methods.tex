This chapter introduces the Sangamon reactors:  the Sangamon200, and a scaled-down version, the Sangamon20; inspired by the pebble bed designs of the PBMR \cite{venter_pbmr_2005, noauthor_pebble_2017} and Xe-100 \cite{harlan_ans_2017, harlan_x-energy_2018}, and the smaller size of the HTR-10.  Both Sangamon200 and Sangamon20 are UCO-pebble fueled, helium cooled reactors.  All simulations used Serpent version 2 \cite{leppanenjaakko_serpent_2015} with postprocessing and analysis performed using Python \cite{van_rossum_python_nodate} and the Python libraries numpy \cite{harris_array_2020} and PyNE \cite{scopatz_pyne:_2012}.  The first sections will lay the foundation for the model, describing particle dispersal and run parameters.  The following sections, \autoref{meth-burn,meth-comp}, detail the single-pebble models, which are used to find the fuel compositions used in the full-core models --- described in \autoref{s200,s20}.  Finally, the last sections of this chapter detail the three tests this work performs using the Sangamon20 model

\section{Modeling Particle and Pebble Dispersal}

Often in HTGR modeling, a uniform lattice is used to approximate the locations of TRISO particles and fuel pebbles.  However, this doesn't reflect realistic TRISO and fuel distributions --- pebbles are not perfectly stacked in the core.  Further, models utilizing lattices often cut-off portions of pebbles at the boundaries that don't perfectly fit.  Instead, the Sangamon200 and Sangamon20 use a random placement of pebbles.  This random dispersal is not only truer to a real pebble-bed, but restricts pebble locations to the boundaries of the core.

In order to determine the locations of random TRISO particles and pebbles, the Serpent particle dispersal routine was leveraged.  It takes the number of particles, defined by the user, or $\eta_{pf}$, the packing fraction (the total volume of particles divided by the volume of that space).  The dispersal routine has the user define the particle radius, and the size and shape of the volume housing the particles.  The routine first randomly determines a single point for each particle contained in the volume.  Then, the routine uses the 'growth factor' and 'shake factor' - both described as fractions of the particle radius, and iterates.  During each iteration, the size of the point's radius increases by the growth factor.  Additionally, the center will move in a random direction a distance equal to the shake factor.  If the particle growth causes the particle to overlap with another particle or leave the volume, it doesn't grow that cycle.  Similarly, if the center's movement causes overlap or the particle to leave the containing volume, it doesn't move.  The dispersal routine iterates until all particles are to their full size, contained in the volume, and not overlapping with any other particles.  The routine generates an output file, in which each line gives the location of a particle center (in x,y,z coordinates), a particle radius, and the name of the particle type, to associate it with the "pbed" (short for pebble bed) card (see the input syntax manual from \cite{leppanenjaakko_serpent_2015}).

\section{Run Parameters and Conditions}
\label{sec-run-params}

All Serpent 2 simulations and post-processing were run on an Ubuntu 18.04 machine using Python version 3.8.5, numpy version 1.19.2, and PyNe version 0.7.1.  Cross-section data is from the JEFF 3.1.2 data libraries.  Below, Table \ref{table:run-params} gives the Monte Carlo run parameters for each Serpent model.

\begin{table}[h!]
\centering
\caption{Reactor Parameters}
\begin{tabular}{ c  c  c  c }
\hline
Parameter & Single Pebble & Sangamon200 & Sangamon20 \\
\hline
Active Cycles & 500 & 150 & 100 \\
Inactive Cycles & 250 & 50 & 50  \\
Neutron Population & 20000 & 70000 & 50000 \\
\hline
\end{tabular}

\label{table:run-params}
\end{table}

All input files are available on github at \cite{richter_zoerichterphlox_2021}.

\section{Burnup and Depletion Methodology}
\label{meth-burn}
In order to determine isotopic compositions in the pebbles, a Serpent burnup simulation of a single pebble ran in burnup steps of 180, 360, 540, 720, 900, and 1080 days --- to represent a number of six-month passes.  The single pebbles are the only simulations that utilize individually defined TRISO particles by default.  Each pebble has an inner region containing the TRISO particles embedded in graphite, and an outer region consisting only of graphite as illustrated by Figure \ref{fig:pebb-zone1}.  The material properties of fresh UCO are from \cite{helmreich_year_2017} and \cite{nagley_fabrication_2010}.  Material properties for TRISO particle layers are from \cite{accuratus_silicon_2013}, \cite{espi_metals_graphite-pyrolytic_2019}, \cite{ho_graphite_1988}, and \cite{johnson_properties_1976}.  The graphite reflector properties are assumed to be the same as the buffer layer in TRISO particles (which is assumed to have the same properties as the graphite matrix in the pebbles).

\begin{table}[h!]
\centering

\caption{Pebble Parameters}
\begin{tabular}{ c  c }
\hline
Parameter & Value \\
\hline
Fueled-Center Radius [cm] & 2.5 \\
Graphite Outer Shell Thickness [cm] & 0.5 \\
Total Radius [cm] & 3.0 \\
TRISO Particles per Pebble & 18,000 \\
\hline
\end{tabular}
\label{table:peb-params}
\end{table}

\begin{figure}[H]
\centering

\includegraphics[width=0.5\linewidth]{figures/pebble-zones.png}
\caption{Pebble Zones}
\label{fig:pebb-zone1}
\end{figure}


Above, Figure \ref{fig:pebb-zone1} shows the fueled vs non-fueled regions of the pebbles, while Table \ref{table:peb-params} gives the full measurements of the pebbles.  For homogenized pebbles, the fuel region is the homogenized center consisting of dispersed TRISO particle material blended with graphite in the region labeled "fuel zone" in Figure \ref{fig:pebb-zone1}, while the non-fueled region is pure graphite.  Each region homogenized by volume fraction uses the "mix" card (see \cite{leppanenjaakko_serpent_2015}) in Serpent to blend the fuel, TRISO layer, and graphite matrix materials.  In heterogenized models, the fueled region marks the part of the pebble where TRISO particles are located inside a graphite matrix, and the non-fuel region is the area of pure graphite that surrounds it.  Below, Figure \ref{fig:particle-layer} and Table \ref{table:particle-params} give detailed measurements on the size of a TRISO particle and its layers, and gives a visual reference.  Figures \ref{fig:particle-layer} and \ref{fig:pebb-zone1} are both to scale.

\begin{figure}[h!]
\centering
\includegraphics[width = 10cm]{figures/trisos-r-like-onions.png}
\caption{TRISO Particle Layers (not to scale)}
\label{fig:particle-layer}
\end{figure}

\section{Fuel Composition}
\label{meth-comp}

The residence time of a pebble in the active core determines its isotopic composition.  We chose to model seven possible pebble compositions, one for each of the six 6-month passes, plus an additional composition for fresh pebbles.  The seven pebble compositions are equally and randomly distributed in the core.

The design approximates the exact isotopic composition by running a burnup calculation using Serpent for a single pebble in a cube with a reflective boundary condition to create an infinite lattice.  The void is filled with helium of the same material properties as in the full core models.  Just as with the location of the pebbles in the full core, the Serpent particle dispersal routine generated the TRISO particle locations.

\begin{figure}[h!]
\centering
\includegraphics[width = 10cm]{figures/burn-20.png}
\caption{Geometry of the Single-Pebble Burnup Calculation: Sangamon1}
\label{fig:burn-20}
\end{figure}

Figure \ref{fig:burn-20} shows the cross-section of the pebble the depletion model is based upon.  Once the depletion simulation determines the isotopic compositions for all six burnup states, the model homogenizes the pebbles by volume, to improve performance.  The volume of a TRISO particle, and more specifically, a UCO kernel, is constant.  The effects of homogenization were explored in the heterogenization tests, which we will describe in \autoref{het-test-meth}.

\section{Sangamon200}
\label{s200}
The Sangamon200 is a 200 MWth helium cooled reactor, with relevant parameters as defined in Table \ref{table:params1}.  Though the model does use some parameters from pre-established designs such as the Xe-100 referenced in Table\ref{table:params1}, it is a simplification to not only reduce computational load, but to create a generic HTGR pebble-bed whose analysis can more broadly apply to pebble-based HTGRs of similar size.  The top and bottom of the reactor core are a flat surface, to create a cylindrical shape for the active core.  The graphite reflector surrounds it with no barriers between the reflector and active core region. These are the only simulated parts of the reactor - there are no control rods included.  In addition, the graphite reflector is a solid cylindrical shell, a container for the pebbles.

\begin{table}[h!]
\centering
\caption{Geometric and Internal Core Parameters in the Sangamon Reactors}
\begin{tabular}{ c  c  c }
\hline
Parameter & Sangamon200 \cite{harlan_x-energy_2018}, \cite{harlan_ans_2017} & Sangamon20 \\
\hline
Thermal Power [MW] & 200 & 20 \\
Average Core Temperature [K] & 800 & 800 \\
Enrichment [wt\%] & 15.5\% & 19.75\% \\
Average Core Pressure [MPa] & 5.9 & 5.9 \\
Outer Core Radius [cm] & 216 & 165 \\
Outer Core Height [cm] & 1150 & 330 \\
Reflector Thickness [cm] & 92 & 75 \\
Number of Pebbles & 220,000 & 22,680 \\
Packing Fraction [\%] & 53.0 & 56.0 \\
\hline
\end{tabular}

\label{table:params1}
\end{table}

While Sangamon200 is not the focus of this assessment, some parameters were used to develop the Sangamon20's design.  We used the outward current of the reflector to constrain the Sangamon20 design.  To do so, a surface detector placed in the reflector, just inside the outer bound of the reflector, shown in Figure \ref{fig:det-place}, tracks the outward neutron current.  When determining the appropriate reflector thickness in Sangamon20, this current is used as an upper boundary and reference point --- that is, the graphite reflector must not only be sufficient to keep the reactor critical, but must also keep the outward surface current less than or equal to Sangamon200's to protect the \acrshort{rpv}.

\begin{figure}[h!]
\centering
\includegraphics{figures/detector-layout.png}
\caption{Detector Placement Inside Reflector}
\label{fig:det-place}
\end{figure}

In Figure \ref{fig:det-place}, the graphite reflector outer boundary is shown, with the surface detector shell shown inside of it.  The active core region, which contains the fuel pebbles, is the innermost region.  The detector is centered on the same point as the reactor as a whole.  Its height is 2 $\left[cm\right]$ less than that of the reflector (e.g., the top of the detector is 1 $\left[cm\right]$ below the top of the reflector) and the radius is 1 $\left[cm\right]$ less than that of the reflector.  This detector measures the outward neutron current in $\left[\frac{neutrons}{s}\right]$.  To arrive at the unit of $\left[\frac{\#}{cm^2s}\right]$ most are familiar with, we divide by the detector's surface area thus:

\begin{align}
J^+ &= \frac{J_s^+ }{S_{d}}
\intertext{where}
J^+&= \mbox{ outward neutron current $\left[\frac{\#}{cm^2s}\right]$}\nonumber\\
J_s^+&= \mbox{ surface unadjusted outward neutron current $\left[\frac{\#}{s}\right]$}\nonumber\\
S_{d}&=\mbox{ detector surface area $[cm^2]$}\nonumber
\end{align}

After accounting for the surface area, the outward current at the detector is $7.351x10^{11} \left[\frac{n}{cm^{2}s}\right]$.

\section{Sangamon20}
\label{s20}

Sangamon20 is a 20 MWth helium-cooled pebble bed reactor, fueled with 19.75\% enriched uranium oxycarbide.  While the thermal power of Sangamon20 is 10\% that of Sangamon200, it isn't sufficient to simply scale Sangamon200's dimensions down to 10\% of their original values, as that wouldn't have the correct volume for the required pebbles, and the neutronics --- such as leakage --- would be inconsistent.  An inner core volume that is 10\% Sangamon200's should be a good approximation, but in order to be certain that this volume would hold the mass of fuel needed, a constraining calculation was carried out, which is detailed in \autoref{r-h-vol}.

\subsection{Graphite Reflector Thickness Determination}

The reflector must be sufficiently thick to keep the reactor critical, and protect the pressure vessel form radiation damage.  To ensure this, we imposed a condition that the outward current in Sangamon20 must be less than or equal to the outward current in Sangamon200 at the outer reflector boundary.  The detector layout in Sangamon20 is identical to Figure \ref{fig:det-place}.

\subsection{Inner Core Volume Determination}
\label{r-h-vol}

The first assumption made in the scale-down is that Sangamon200 and Sangamon20 have the same specific power, or $\left[ \frac{\text{kW}}{\text{g UCO}} \right]$.

To calculate the mass of fuel in Sangamon200:


\begin{align}
M_{f,200} &= \frac{4}{3}\pi r_{u}^3 \rho_{u} n_{T} n_{p,200} \label{mf200}
\intertext{where}
M_{f,200}&= \mbox{ mass of fuel in Sangamon200 $\left[g\right]$}\nonumber\\
r_{u}&= \mbox{the radius of the UCO kernel inside a TRISO particle $\left[cm\right]$}\nonumber\\
\rho_{u}&= \mbox{ the density of UCO in $\left[\frac{g}{cc}\right]$}\nonumber\\
n_{T}&= \mbox{ number of TRISO particles in one pebble}\nonumber\\
n_{p}&= \mbox{ number of pebbles in Sangamon200}\nonumber
\end{align}


Using the parameters from Table \ref{table:params1}, the power density of Sangamon200 and Sangamon20 is 0.11 $[\frac{kW}{g}]$.  With a power capacity of 20 MWth, one can calculate the total mass of UCO in Sangamon20 as

\begin{align}
M_{f,20} &= \frac{P}{\rho_{p}} = 181818.18 \left[g\right]
\intertext{where}
M_{f,20}&= \mbox{ total mass of UCO in Sangamon20 [g]}\nonumber\\
P&= \mbox{ Thermal power of Sangamon20 [kW] }\nonumber\\
\rho_p &=\mbox{ Sangamon20's power density $[\frac{kW}{g}]$}\nonumber
\end{align}

Equation \ref{mf200} calculates the total mass of fuel in the Sangamon200 reactor by first calculating the mass of UCO in a single pebble using the density of UCO and the total volume of UCO kernels in a single pebble.  This value is then multiplied by the number of pebbles in Sangamon 200 (see Table \ref{table:params1}).  The total mass of fuel in the reactor divided by the mass of fuel in a single pebble gives the number of pebbles in the reactor, as follows:

\begin{align}
n_{p,20} &= \frac{M_{f,20}}{\frac{4}{3}r_{u}^3n_{T}\rho_{u}}
\intertext{where}
n_{p,20}&= \mbox{ number of pebbles in Sangamon20 [-]}\nonumber\\
M_{f,20}&= \mbox{ total mass of UCO in Sangamon20 [g]}\nonumber\\
r_u&=\mbox{ radius of a UCO kernel [cm]}\nonumber\\
n_{T}&= \mbox{ number of TRISO particles in a single pebble [-]}\nonumber\\
\rho_{u}&= \mbox{ density of UCO $[\frac{g}{cm^3}]$}\nonumber
\end{align}
\\
Rounding up - there can only be complete pebbles - we arrive at the number of pebbles for the Sangamon20 as shown in Table \ref{table:params1}.

Knowing the number of pebbles is insufficient - the exact dimensions of the active core region are still undefined.  To determine the volume of this space, the formula uses concept of the packing fraction.  The packing of even uniform objects in a 3-dimensional space is a complicated problem \cite{tulluri_analysis_nodate}.  Assuming the pebble behavior is random loose packing \cite{tulluri_analysis_nodate} - the pebbles have unsystematically fallen into the core and the core is unshaken - the packing fraction in the range of 0.56 to 0.60 \cite{tulluri_analysis_nodate}.  Using the definitions above, the active core volume is

\begin{align}
V_{c,20} &= \frac{ n_{p,20}\frac{4}{3}\pi r_{p}^3 }{ \eta_{pf} }
\intertext{where}
V_{c,20}&= \mbox{ volume of the active core in Sangamon20 $[cm^2]$}\nonumber\\
n_{p,20}&= \mbox{ number of pebbles in Sangamon20 [-]}\nonumber\\
r_p&=\mbox{ radius of a pebble [cm]}\nonumber\\
\eta_{pf}&= \mbox{ packing fraction [-]}\nonumber
\end{align}

Using the formula for the volume of a cylinder, one can plot possible sets of $r_{c,20}$ and $h_{c,20}$ that satisfy the volume requirement.

\begin{figure}[H]
\centering
\includegraphics[width = 10cm]{figures/act-core-RH.png}
\caption{Curve of possible height and radii that satisfy the volume requirements imposed by packing fraction(s) $\eta_{pf}$}
\label{fig:rh-vol}
\end{figure}

The most critical configurations for a cylinder are either a \emph{square} shape, in which the height is equal to the diameter, or a \emph{flat} shape in which diameter is significantly greater than height.  As most designs use a height that is greater than or equal to the diameter, Sangamon20 is a square cylinder.  The point indicated in \ref{fig:rh-vol} shows the radius and height selected for Sangamon20 - a radius of 90 $\left[cm\right]$, and a height of 180 $\left[cm\right]$.


\section{Heterogenization Tests}
\label{het-test-meth}

As described above, the pebbles use the approximation of a homogenized 'fueled-center', to reduce computational load.  This can, however, make the results of the simulation less accurate.  To test the degree to which this affects reactor parameters such as $k_{eff}$ or outward neutron current, a few tests performed undid this change, explicitly defining all TRISO particles in the pebbles, as they are in the single-pebble (infinite lattice) depletion models which generated the equilibrium fuel composition.  These so-called heterogenization tests compared the 2-group fast and thermal fluxes in the radial and axial direction.  In addition, they compared the lethargy-adjusted neutron energy spectrum using a 315-group energy structure.  Beyond showing the full curves for each of these values for comparison, the relative difference is provided graphically for each.  Other than the choice to explicitly model the TRISO particles, the heterogenized model is identical to the Sangamon20 homogenized-pebble model.

\section{Reactor Sensitivity to Pebble Locations and Symmetry}
\label{meth-sens}

By the nature of random pebble dispersal, there are a theoretically infinite number of perturbations to the Sangamon reactor models that are no different sans slight variations in pebble locations.  It is also entirely possible to have bands in the reactor such that multiple pebbles of same (or similar) burnup form lines or pockets.  In the interest of better characterizing the neutronics of this reactor model, we created a test to explore the effect of various pebble composition locations.  The \emph{shuffling} test maintained the pebble locations, but changed what composition the individual pebbles were (for results, see \autoref{res-shuff}).  The second analyzed the effects of utilizing a symmetry simplification, in order to improve computational speed (see \autoref{res-sym}) where $\frac{1}{6}$ of the core is modeled with a periodic boundary condition (neutrons exiting one side of the symmetry slice enter the other at the same height and trajectory).  The slice used to simplify changed in each test, shown in Figure \ref{fig:slicetest}.  In each test, all other parameters remain the same.

\begin{figure}[h!]
\centering
\includegraphics{figures/run-layout.png}
\caption{Symmetry Test Run Layouts}
\label{fig:slicetest}
\end{figure}


The shuffle tests change which fuel composition is in which pebble.  As an example, Run 1 in the shuffle test makes all fresh, or "zero-pass" pebbles of the first-pass composition, first-pass pebbles of the second-pass, and so on down the line.  Run 2 makes the originally fresh (zero-pass) pebbles the second pass composition, the first-pass pebbles the third-pass composition, and so on. This scheme is described in \ref{table:shuffle}.

\begin{table}[h!]
\centering
\caption{Shuffle Test Run Schemes}
\begin{tabular}{ c  c  c  c  c  c  c }
\hline
Original Fuel Position in Control & Run 1 & Run 2 & Run 3 & Run 4 & Run 5 & Run 6  \\
\hline
0 & 1 & 2 & 3 & 4 & 5 & 6 \\
1 & 2 & 3 & 4 & 5 & 6 & 0 \\
2 & 3 & 4 & 5 & 6 & 0 & 1 \\
3 & 4 & 5 & 6 & 0 & 1 & 2 \\
4 & 5 & 6 & 0 & 1 & 2 & 3 \\
5 & 6 & 0 & 1 & 2 & 3 & 4 \\
6 & 0 & 1 & 2 & 3 & 4 & 5 \\
\hline
\end{tabular}

\label{table:shuffle}
\end{table}

In both tests, the $k_{eff}$ and outward currents are recorded and compared to the control model's $k_{eff}$ and $J^+$ (see \autoref{res-control}).  A full spectrum of core components wasn't performed, as the difference between the results of this test and the control were small compared to the heterogeneity tests.  However, a full investigation of neutron energy spectra and core fluxes may be a a potential avenue for future work.  The appendix contains all geometry and mesh images for each run, and a pixel-by-pixel image difference of the mesh results (see \autoref{app}).