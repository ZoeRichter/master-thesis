%% Package and Class "uiucthesis2014" for use with LaTeX2e.
\documentclass[edeposit,fullpage]{uiucthesis2018}

%\usepackage[utf8]{inputenc}


\usepackage[acronym,toc]{glossaries}
\makeglossaries
\newacronym{htgr}{HTGR}{High Temperature Gas-Cooled Reactor}
\newacronym{lwr}{LWR}{Light Water Reactor}
\newacronym{triso}{TRISO}{Tri-Structural Isotropic}
\newacronym{biso}{BISO}{Bi-Structural Isotropic}
\newacronym{smr}{SMR}{Small Modular Reactor}
\newacronym{htg-smr}{HTG-SMR}{High Temperature Gas-Cooled Small Modular Reactor}
\newacronym{rpv}{RPV}{Reactor Pressure Vessel}
\newacronym{sbo}{SBO}{Station Black-Out}
\newacronym{ornl}{ORNL}{Oak Ridge national Laboratory}
\newacronym{avr}{AVR}{Arbeitsgemeinschaft Versuchsreaktor}
\newacronym{leu}{LEU}{Low-Enriched Uranium}
\newacronym{csg}{CSG}{Constructive Solid Geometry}
\newacronym{endf}{ENDF}{Evaluated Nuclear Data Format}
\newacronym{ace}{ACE}{A Compact ENDF}
\newacronym{mcnp}{MCNP}{Monte Carlo N-Particle transport}
\newacronym{beau}{BEAU}{Burnup Equilibrium Analysis Utility}
\newacronym{bct}{BCT}{Body Centered Tetragonal}
\newacronym{hcp}{HCP}{Hexagonal Close Packed}
\newacronym{inl}{INL}{Idaho National Laboratory}
\newacronym{vsop}{VSOP}{Very Superior Old Programs}
\newacronym{pb-fhr}{PB-FHR}{Pebble-Bed Fluoride High Temperature Reactor}
\newacronym{fcc}{FCC}{Face Centered Cubic}
\newacronym{bcc}{BCC}{Body Centered Cubic}
\newacronym{mol}{MOL}{Middle Of Life}
\newacronym{pbmr}{PBMR}{Pebble Bed Modular Reactor}
\newacronym{htr}{HTR}{High Temperature Reactor}
\newacronym{sc}{SC}{Simple Cubic}
\newacronym{efpd}{EFPD}{Effective Full Power Days}
\newacronym{ngnp}{NGNP}{Next Generation Nuclear Plant}
\newacronym{rmc}{RMC}{Reactor Monte Carlo}
\newacronym{rsa}{RSA}{Random Sequential Addition}
\newacronym{dem}{DEM}{Discrete Element Method}
\newacronym{rug}{RUG}{Random Universe Geometry}
\newacronym{uco}{UCO}{Uranium Oxycarbide}
\newacronym{otto}{OTTO}{Once Through Then Out}
\newacronym{rgb}{RGB}{Red-Green-Blue}
\newacronym{vhtrc}{VHTRC}{Very High Temperature Reactor Critical Assembly}
\newacronym{iaea}{IAEA}{International Atomic Energy Agency}
\newacronym{crp}{CRP}{Coordinated Research Project}
\newacronym{uam}{UAM}{Uncertainty Analysis in Modeling}
\newacronym{gif}{GIF}{GEN IV International Forum}


\usepackage{xspace}
\usepackage{graphics}
\newcommand{\Cycamore}{\textsc{Cycamore}\xspace}
\newcommand{\Cyclus}{\textsc{Cyclus}\xspace}


\usepackage{placeins}
\usepackage{booktabs} % nice rules (thick lines) for tables
\usepackage{microtype} % improves typography for PDF

\usepackage[hyphens]{url}
\usepackage[hidelinks]{hyperref}
%\usepackage{subfig}
\usepackage{hhline}
\usepackage{amsmath}
\usepackage{color}
\usepackage{float}
\usepackage{multirow}
\usepackage{siunitx}
\usepackage{subcaption}
\usepackage{adjustbox}
\sisetup{ 
    input-decimal-markers = .,input-ignore = {,},table-number-alignment = right,
    group-separator={,}, group-four-digits = true
}
\usepackage{fourier}
\usepackage{booktabs}
\newcommand\tab[1][1cm]{\hspace*{#1}}

\usepackage{threeparttable, tablefootnote}

%tikzpicture fit to page width
\usepackage{environ}
\makeatletter
\newsavebox{\measure@tikzpicture}
\NewEnviron{scaletikzpicturetowidth}[1]{%
  \def\tikz@width{#1}%
  \def\tikzscale{1}\begin{lrbox}{\measure@tikzpicture}%
  \BODY
  \end{lrbox}
  \pgfmathparse{#1/\wd\measure@tikzpicture}%

  \edef\tikzscale{\pgfmathresult}%
  \BODY
}

\usepackage{tabularx}
\newcolumntype{b}{>{\hsize=1.0\hsize}X}
\newcolumntype{q}{>{\hsize=0.5\hsize}X}
\newcolumntype{R}{>{\raggedleft\arraybackslash\hsize=0.5\hsize}X}
\newcolumntype{z}{>{\hsize=0.75\hsize}X}
\newcolumntype{s}{>{\hsize=.5\hsize}X}
\newcolumntype{m}{>{\hsize=.75\hsize}X}

\usepackage{longtable}

\usepackage{cleveref}
\usepackage{datatool}
\usepackage[numbers]{natbib}
\usepackage{notoccite}

\usepackage{tikz}
\usetikzlibrary{positioning, arrows, decorations, shapes, shadings}

\usetikzlibrary{shapes.geometric,arrows}
\tikzstyle{process} = [rectangle, rounded corners, minimum width=2.5cm, minimum height=1cm,text centered, draw=black, fill=blue!30]

\tikzstyle{object} = [ellipse, rounded corners, minimum width=3cm, minimum height=1cm,text centered, draw=black, fill=green!30]
\tikzstyle{objectr} = [ellipse, rounded corners, minimum width=3cm, minimum height=1cm,text centered, draw=black, fill=red!30]

\tikzstyle{empty} =  [rectangle, rounded corners, minimum width=2.5cm, minimum height=0.7cm,text centered, draw=black, fill=white!30]
\tikzstyle{arrow} = [thick,->,>=stealth]

\newcommand\mainmatterWithoutReset
 {\edef\temppagenumber{\arabic{page}}%
  \mainmatter
  \setcounter{page}{\temppagenumber}%
 }

\title{Isotopic and Reactor Physics Characterization of a Gas-Cooled, Pebble-Bed Microreactor}
\author{Zo{\"e} Richter}
\department{Nuclear, Plasma, Radiological Engineering}
\schools{B.S., University of Illinois - Urbana Champaign, 2018}
\msthesis
\advisor{Madicken Munk}
\degreeyear{2021}
\committee{Dr. Madicken Munk \\ Dr. Tomasz Kozlowski}

% Template created by Charles Bae, 2018

\begin{document}
\maketitle

\frontmatter
%% Create an abstract that can also be used for the ProQuest abstract.
%% Note that ProQuest truncates their abstracts at 350 words.
\begin{abstract}
Pebble-bed \acrfull{htgr} designs present a unique modeling challenge.  Pebble-beds can have a variety of pebble compositions due to varying levels of burnup.  In addition, the pebbles are mobile in the core --- entering from the top and exiting through the bottom --- and may fall in a haphazard arrangement.  This work introduces a 20 MWth pebble-bed HTGR reactor design (which will be referred to as Sangamon20) and investigates not only the neutronics of the base model, but the changes to core neutronics after making modifications to the simulation.  These modifications include: heterogenizing versus homogenizing the pebble center, imposing a universal symmetry assumption, and changing the arrangement of pebble fuel compositions, all using Serpent and Python.  All of this is in support of the ultimate goal of this project --- to establish a baseline source term and to determine what simplifications, if any, can be made in the model without sacrificing accuracy in the results.  This is crucial for any future licensing effort, safety analysis, or accident analysis involving pebble-bed reactors.  This model of Sangamon20 uses a random dispersal of seven different pebble compositions, each corresponding to a different burnup level.  The heterogenization tests compare $k_{eff}$, thermal and fast flux profiles, and the neutron lethargy-adjusted energy spectra in the core, reflector, coolant, a random fresh, and a random discharge-burnup pebble.  Shuffling and symmetry tests monitor changes to $k_{eff}$ and the outward neutron current at the outer reflector boundary; the latter because it can be used to find the anticipated neutron flux the \acrfull{rpv} would experience.  This informs the level of radiation damage one could expect the \acrshort{rpv} to experience each year - which is useful from a design and safety perspective.  Neither the symmetry test nor the shuffling test caused a major difference in either the $k_{eff}$ or the outer-bounds outward neutron current.  This would suggest that there is no need to simulate all possible pebble placements to characterize a reactor.  However, for the heterogenization tests, $k_{eff}$ differed by over 4.0\%, and the pebble spectra at certain higher energies disagreed by a factor of 2-4.  A complete fuel isotopic composition is accessible at \cite{richter_isotopic_2021}, and this thesis discusses select isotopic inventories.  These can be used to inform source term determination or spent fuel compositions.

\end{abstract}

\chapter*{Acknowledgments}
I'd like to thank my advisor, Dr. Kathryn Huff, for offering her knowledge and support through these trying times and uncertainties.  I would also like to thank Dr. Madicken Munk, whose insight and experience was invaluable.  To Nathan Ryan, my tireless reviewer: thank you for all of your hard work.  I would like to thank everyone in the Advanced Reactors and Fuel Cycles group, past and present, for giving me their time, aid, and experience: Amanda Bachmann, Dr. Andrei Rykhlevskii, Ansh Chaube, Greg Westphal, Gwendolyn Chee, Mehmet Turkmen, Roberto Fairhurst, Sam Dotson, and Sun Myung Park.  

Thank you to my family, for always believing in me; Trinket, for loving hugs; and Sara, without whom nothing in this project would have a good name.


This work is supported by the Nuclear Regulatory Commission Fellowship Program.

%% The thesis format requires the Table of Contents to come
%% before any other major sections, all of these sections after
%% the Table of Contents must be listed therein (i.e., use \chapter,
%% not \chapter*).  Common sections to have between the Table of
%% Contents and the main text are:
%%
%% List of Tables
%% List of Figures
%% List Symbols and/or Abbreviations
%% etc.

\tableofcontents
\listoftables
\listoffigures
\addcontentsline{toc}{chapter}{List of Abbreviations}

%% Create a List of Abbreviations. The left column
%% is 1 inch wide and left-justified
\chapter*{List of Abbreviations}
\begin{longtable}{p{25mm} p{50mm} rp{75mm}}
\acrshort{ace} & & \acrlong{ace} \\
\acrshort{avr} & & \acrlong{avr} \\
\acrshort{beau} & & \acrlong{beau} \\
\acrshort{bcc} & & \acrlong{bcc} \\
\acrshort{bct} & & \acrlong{bct} \\
\acrshort{biso} & & \acrlong{biso} \\
\acrshort{crp} & & \acrlong{crp} \\
\acrshort{csg} & & \acrlong{csg} \\
\acrshort{dem} & & \acrlong{dem} \\
\acrshort{efpd} & & \acrlong{efpd} \\
\acrshort{endf} & & \acrlong{endf} \\
\acrshort{fcc} & & \acrlong{fcc} \\
\acrshort{gif} & & \acrlong{gif} \\
\acrshort{hcp} & & \acrlong{hcp} \\
\acrshort{htg-smr} & & \acrlong{htg-smr} \\
\acrshort{htgr} & & \acrlong{htgr} \\
\acrshort{htr} & & \acrlong{htr} \\
\acrshort{httr} & & \acrlong{httr} \\
\acrshort{iaea} & & \acrlong{iaea} \\
\acrshort{inl} & & \acrlong{inl} \\
\acrshort{leu} & & \acrlong{leu} \\
\acrshort{lwr} & & \acrlong{lwr} \\
\acrshort{mcnp} & & \acrlong{mcnp} \\
\acrshort{mol} & & \acrlong{mol} \\
\acrshort{ngnp} & & \acrlong{ngnp} \\
\acrshort{ornl} & & \acrlong{ornl} \\
\acrshort{otto} & & \acrlong{otto} \\
\acrshort{pb-fhr} & & \acrlong{pb-fhr} \\
\acrshort{pbmr} & & \acrlong{pbmr} \\
\acrshort{rgb} & & \acrlong{rgb} \\
\acrshort{rmc} & & \acrlong{rmc} \\
\acrshort{rpv} & & \acrlong{rpv} \\
\acrshort{rsa} & & \acrlong{rsa} \\
\acrshort{rug} & & \acrlong{rug} \\
\acrshort{sbo} & & \acrlong{sbo} \\
\acrshort{sc} & & \acrlong{sc} \\
\acrshort{smr} & & \acrlong{smr} \\
\acrshort{thtr} & & \acrlong{thtr} \\
\acrshort{triso} & & \acrlong{triso} \\
\acrshort{uam} & & \acrlong{uam} \\
\acrshort{uco} & & \acrlong{uco} \\
\acrshort{vhtrc} & & \acrlong{vhtrc} \\
\acrshort{vsop} & & \acrlong{vsop}
\end{longtable}

%\printglossary
%\printglossary[type=\acronymtype, title=List of Acronyms, toctitle=List of Acronyms]
%% Create a List of Symbols. The left column
%% is 0.7 inch wide and centered

\pagebreak
\mainmatter

\chapter{Introduction}
outline:
\begin{itemize}
\item Motivation
	\begin{itemize}
	\item climate change
	\item gen IV > gen III
	\item microreactors alleviate many of the construction/installation issues associated with larger reactors
	\item pebbles using TRISO particles are an ideal fuel form, if there are no plans to reprocess (the US currently does not)
	\end{itemize}
\item Objectives
	\begin{itemize}
	\item establish a basic model for a HTGR pebble-bed microreactor
	\item characterize the effects of random pebble placement on the results
	\item describe a scale-down from a basic 200 MWth design to a 20 MWth microreactor
	\end{itemize}
\end{itemize}




\section{Motivation}

The effects of global warming are becoming increasingly severe. (***include NASA facts?***).  In the interest of reducing the global carbon footprint, a desire for carbon-free, sustainable energy is growing. With this interest comes a bevy of new research in the next generation of nuclear reactors.

One such class of reactors are the high temperature gas-cooled reactor, or HTGR.  While HTGRs can have a variety of fuel forms, of particular interest are pebble bed reactors.  A pebble-type fuel generally consists of a sphere of graphite, approximately the size of a billiard ball, embedded with TRISO particles.  The fuel kernels in these TRISO particles are surrounded in multiple layers of carbon and silicon carbide, and, along with the graphite that creates the sphere proper, form a durable, compact fuel form.  In addition, the pebbles are able to be refueled online, reducing the need for planned shutdowns.

The next generation of nuclear reactors also include designs significantly smaller than the conventional Light Water Reactor(LWR) seen in the USA today.  So-called Small Modular Reactors, or SMRs, these reactors are small enough to be shipped, reactor pressure vessel and all, in a standard shipping truck or train.  The pressure vessels can also be produced in a factory of standard size (**** I dislike this wording. but can't think of another way to put it****).  SMRs can be deployed in a variety of new settings, such as isolated towns or work sites, or many can be stationed together in one plant to fill the role of a single larger reactor.

This work used a pebble-bed HTG-SMR as a starting point, and modeled a fairly generic 200MWth reactor based on existing designs - named Sangamon200. Then it scaled down to a target size - a 20MWth pebble bed HTGR.  "Microreactors" such as these are generally 70 MWth or less, and can be deployed in areas where only a small amount of power is needed, used for research and testing, or be used to supply heat for other industrial processes, such as producing hydrogen.

The 20MWth model, which will hereafter be referred to as Sangamon20, is of a highly simplified design, which can be used in future testing and analysis.

*******why modular: reutler and lohnert**********

\section{Objectives}



\chapter{Literature Review}
***(General notes/questions
\begin{itemize}
\item I wasn't sure how much to include about the Cisneros thesis and BEAU.  It's true that it informed how I handled the pebbles - knowing they move in slug flow, for example - but I didn't use any of the lattice models that thesis did, and while I would like to explore the fuel burnup a bit more in depth next, I'm not sure that fits this thesis right now
\item I think we talked about the fact that i can afford to go into more detail on PBMR because of the long time spent on it, but I'll wait for the first review to add anything
\item I was going to include a brief section on some micro reactor designs from recent years, but I don't really have a set method for selecting which designs and articles I should mention, and I don't want it to seem random or hodge-podge.  (if I do make a few paragraphs about this, would it be appropriate to reference Roberto's thesis? technically I think it definitely fits, I mean more in the context of being polite/proper)
\end{itemize}
)***
\section{Computational Model}

\subsection{Serpent}

Serpent 2 is "a multi-purpose three-dimensional continuous-energy Monte Carlo particle transport code" \cite{noauthor_serpent_nodate} from the VTT Technical Research Center of Finland.  The first iteration, Serpent 1, began development in 2004.  The development of Serpent 2 is presently ongoing.  Serpent 2 has three main applications: traditional reactor physics, coupled multi-physics, and neutron and photon transport.  In order to create and model complex geometries, Serpent uses constructive solid geometry (CSG), which defines homogeneous material cells using user-defined universes, cells, lattices, and specially-defined nested objects to define particle and pebble geometries.  

Using these special objects and the particle dispersal routine in Serpent, TRISO particles and pebble bed reactors can be modeled.  The particle dispersal routine works by first establishing a user defined container for the particles, such as a cylinder.  The user then defines the size of the particles - all of which are assumed to be perfect spheres - and either the number of particles, or the packing fraction of particles in the container.  Serpent then randomly generates coordinates for the center of the dispersed particles.  For each location, Serpent checks that the entire particle is inside the container, and that it does not overlap with any other particles.  It then prints a text file with the coordinates of each particle center, along with the radii and the universe they're located in.  Serpent has been tested with up to 60 million individual particles.

Physics are based on a combination of classical kinematics, ENDF reaction laws, and random sampling.  For particle transport, Serpent uses surface tracking and Woodcock-delta tracking.  For material data, Serpent uses ACE format libraries for microscopic cross sections, and pre-generates macroscopic cross sections before beginning transport.  To further speed-up calculations, Serpent uses a unionized energy grid.

Serpent has been validated against MCNP, and validation is ongoing for radiation shielding and criticality safety analysis.  While the differences between Serpent and other Monte Carlo codes are generally small, Serpent experiences the same issues validating its results as other Monte Carlo programs, related to small differences in data libraries.

\subsection{Experience from BEAU}

BEAU, or Burnup Equilibrium Analysis Utility, was developed by Dr.Cisneros to model depletion and multiple burnup states for a continuously refueled molten-salt PBMR.  While a full examination of all possible pebble burnups and compositions is not the target of this work, the experience in modeling a pebble bed reactor was invaluable.  In particular, the development of the Mark 1 PB-FHR (pebble-bed fluoride salt cooled high temperature reactor), the methods to model pebbles, and the flow of pebbles through the core, informed the Sangamon20 design.


\section{The High Temperature Gas Cooled Reactor: Beginnings and Concepts}

High temperature gas cooled reactors, or HTGRs, are one of the more commonly seen Generation IV reactor designs.  It most often uses helium as a coolant, and graphite as a moderator in thermal designs.  Fuel is in the form of tristructural-isotropic, or TRISO particles.  TRISO particles use a small kernel of fuel, less than half a millimeter across, surrounded by layers of carbon and silicon carbide to protect the fuel kernel and prevent the leakage of radioisotopes ***(would it also be accurate to say the layers provide moderating material?  I know that graphite is a great moderator, but I would think the main purpose is safety and to prevent leaching, and moderation is really in the graphite the particles are embedded in)***.  These TRISO particles are then embedded in graphite to form a usable fuel element.  In prismatic HTGRs, the graphite is in the shape of hexagonal columns.  In pebble-bed reactors, the graphite is in the shape of spheres, around the size of a billiard ball.  Many of these pebbles are loaded into the core, and slowly move through the bottom in a manner not dissimilar to grain in silos.

HTGRs, however, are not a new concept.  Preliminary concepts for a gas-cooled reactor were created as early as 1942.  Farrington Daniels - more commonly known for work in chemistry and solar power technology - is attributed with establishing the first theoretical designs.  A professor from the University of Wisconsin, Professor Daniel's work with Oak Ridge National Laboratory (ORNL) nailed-down the most basic characteristics of the HTGR.  The choice of helium for coolant, graphite for moderator, the direct gas turbine cycle, and the use of uranium or thorium carbides for fuel all came from his work \cite{simnad_early_1991}.

Professor Daniels recognized early on the importance of a small power-producing unit with low initial costs and ease of transport in developing nations - a sentiment that has resurfaced in recent years in the designs for small modular reactors.  Daniels' first design was for a simple high-temperature pebble pile.  A little over a decade later, improvements to turbine technology prompted him to propose a direct-cycle helium cooled reactor.  However, before any construction could start, the Atomic Energy Commission opted to support Light Water Reactor (LWR) technology instead \cite{simnad_early_1991}.

\begin{figure}[h!]
\centering
\includegraphics[width=0.6\linewidth]{figures/daniels-1}
\caption{Side-View of the 1955 Daniels' Concept, \cite{simnad_early_1991}}
\label{fig:daniels-1}
\end{figure}
\begin{figure}[h!]
\centering
\includegraphics[width=0.6\linewidth]{figures/daniels-2}
\caption{Diagram of Coolant Flow in the 1955 Daniel's Concept, \cite{simnad_early_1991}}
\label{fig:daniels-2}
\end{figure}

Figures \ref{fig:daniels-1} and \ref{fig:daniels-2} show the design of the 1955 design proposed by Professor Daniels.  Like many modern modular reactor plant designs, Professor Daniels suggested that the reactor be mostly underground.  A key difference between the Farrington Daniels designs and modern HTGRs is the fuel form.  While modern designs use TRISO particles embedded in graphite, the Daniels' design uses solid graphite blocks, with channels for both coolant and fuel.  Within the fuel channels, fuel was loaded in either a pellet or cartridge form, both a mixture of 10$\%$ uranium dicarbide and graphite powder.  In addition to these fuel channels, the design included an outer ring of graphite reflector in which thorium was used to breed U-233.  Control rods were made of boron-containing molybdenum.  Additional safety rods made of the same were held above the core by steel wires that would melt in the case of an accident, dropping the safety rods into the core \cite{simnad_early_1991}.

\section{Earliest Operational HTGRs}

The earliest operational HTGRs were first started in the 1960s.  The AVR, from Germany, Dragon, operating in the UK, and Peach Bottom 1, which operated in the US \cite{beck_high_nodate}.

\subsection{Dragon}

The Dragon prismatic HTGR was a test reactor operated in Winfrith, UK, from 1964 to 1975, making it the oldest of the reactors discussed in this chapter.  It operated at inlet and outlet temperatures of 350\textdegree  C and 750\textdegree  C and a power of 20MWt \cite{beck_high_nodate}.  Dragon's main purpose was to test reactor materials, with an emphasis on fuels.  It originally used uranium and thorium as fuel, but switched to a purely uranium-based fuel with a lower enrichment later in life.  The fuel elements themselves were similar in shape to the Daniels' design - hexagonal prisms with fuel rod channels.

Contrary to the fuel philosophy seen today, Dragon originally allowed fission products to be released from fuel elements into the circulating helium coolant.  The fission products would then be purged from the helium.  However, Dragon later switched to a coated-particle fuel when it became clear that having such large fission product releases would be difficult to manage \cite{simnad_early_1991}.

\subsection{Peach Bottom 1}

Peach Bottom 1 operated from 1966 to 1974, by the Philadelphia Electric Company.  It was the first operational HTGR in the US, and the first to produce electric power.  It was slightly larger than Dragon, at a nameplate capacity of 115 MWt/40MWe and a slightly lower operating temperature range at 327\textdegree  C to 700\textdegree  C inlet to outlet \cite{beck_high_nodate}.  Like Dragon, Peach Bottom 1 was a prismatic reactor, however, Peach Bottom used coated uranium and thorium carbide particles from the beginning.  The original fuel used a single coating of pyrolytic carbon.  However, after multiple fuel failures, Peach Bottom upgraded to bistructural isotropic, or BISO, fuels by adding an additional layer.  Peach Bottom would later upgrade the fuel once again by adding a silicon carbide layer, forming TRISO particles \cite{beck_high_nodate}.  One operational benefit of upgrading to TRISO particles from BISO particles was that the superior fission product retention meant that Peach Bottom 1 could remove the helium purging systems.  In addition to the inner fuel region, Peach Bottom, like the Daniels' design, bred U-233 in an outer region using thorium.

Beyond changing the number and materials for fuel coatings, the experiences in Peach Bottom 1 helped to develop HTGR fuel elements.  Operators saw that by using the graphite moderating material to dilute the fuel, the fuel could be diluted further compared to other diluents.  This of course has the advantage of saving fuel material, but also improved heat transfer and reduced radiation damage.  Additionally, operational experience showed that, in order to prevent the creation and buildup of U-236 and Np-237, which are poisons, the U-235 and U-233 should be kept separate \cite{simnad_early_1991}.

In the end, Peach Bottom 1 closed when it was determined to be uneconomical.

\subsection{AVR}

The Arbeitsgemeinschaft Versuchsreaktor (AVR) was an experimental pebble-bed reactor operated in the Jülich Research Center from 1967 to 1988.  It had a capacity of 46 MWt/15MWe, with inlet and outlet temperatures of 275\textdegree  C and 950\textdegree  C \cite{beck_high_nodate}.  In fact, the AVR reached the highest operating temperatures of any commercial nuclear plant.  Like the others in this early time period. the AVR used a combination of uranium and thorium fuels, though the AVR began with BISO particles.  The core held around 100,000 graphite pebbles, almost a third of which had fuel in them.

Despite not being built for experimental purposes, the AVR still housed many experiments that improved our body of knowledge in HTGR technology.  During the first few years of its life, the goal of the AVR was to demonstrate that it was a reliable technology.  After this inital period, the AVR could shift to allowing various experiments.

A step was to show that the reactor could operate safely, could control the core power and temperatures and safely shut down and remain sub-critical for long periods of time.  This proved to be quite the undertaking, as the AVR shifted from highly enriched to low enriched fuel over time, which caused a large variety in fuel pebble compositions, on top of the range of compositions inherent to a multi-pass pebble cycle.

The AVR also provided data to validate models of pebble-bed reactors, and conducted an experiment to better characterize the radial distribution of temperatures in the core.  A number of marked pebbles were loaded into the core, each housing a series of wires that would melt at a certain temperature, the lowest being 655\textdegree  C, the highest 1280\textdegree  C.  The pebble positions were tracked based on pebble flow data, and when the spheres were ejected, the were examined to determine what temperatures the pebbles had experienced.  Despite the outlet temperature being determined to be 950\textdegree  C, multiple pebbles experienced a temperature greater than or equal to the 1280\textdegree C maximum temperature in the melt wires.  It was noted that these pebbles went through a zone with a spike in local power density \cite{noauthor_results_1990}.

The AVR also demonstrated the inherent safety of HTGR reactors in accident scenarios by purposefully causing failure of active cooling system "accidents".  In the first, the the coolant blowers were shutoff, and no shutdown rods were inserted while operating at full power.  The operators additionally shut the main circuit valves to prevent natural circulation to regions outside the active core.  Overall, the changes to core temperatures were unremarkable.  The hottest regions cooled, while the coldest regions warmed up.  Additionally, due to negative temperature feedback coefficients, the reactor power immediately declined in response to the "accident".  The temperature slowly rose to 2 MW again over 24 hours, only to level out around 300 kW.  A further test provided data on loss of coolant and depressurization accidents.  As before, the core temperature changes were not particularly drastic.  The upper core region was seen cooling, while the lower, originally cooler core region slowly rose in temperature.  This experiment's data was used to validate HTGR computer models, which allowed the results to be aid in the analysis of other HTGRs \cite{noauthor_results_1990}.

Beyond accident safety, the AVR allowed for testing and demonstration of the safety qualities of TRISO and BISO fuel elements, especially relating to high temperature tolerance and fission product retention.  Inital tests were conducted with BISO based pebbles, then later transitioned to TRISO, then low-enriched TRISO pebbles.  The TRISO-LEU pebbles were shown to have good fission product retention compared to their BISO-based predecessors, based on the results of sampling the activity of the circulating helium to to the presence of released fission products.  Beyond radioisotopes being directly released into the coolant gas, the AVR also showed that in order to accurately characterize the source term of am HTGR pebble bed reactor, one must take the dust from the pebbles into account.  Dust from the pebbles bumping and scraping against each other was found deposited on reactor surfaces in the primary loop.  It was found that 60 kg of dust had accumulated by the end of the reactor's life, which averages to 3 kg of dust each year.  Measurements of specific activity in the dust showed that the activities of Cs-137, Cs-134, I-131, Sr-90, and Co-60 were on the order of $10^6 \frac{Bq}{g}$.  Even though there is relatively little dust, the activity of this dust is fairly high, especially compared to the activity of the coolant gas \cite{noauthor_results_1990}.

***(I'm going to include here the two tables comparing the activities of the coolant gas and dust activities.  However, the one for gas is (reasonably) by volume, while the dust is by mass.  Should I use the density of helium at operating temperature to convert the gas activity to be in Bq/g, and make a new table (citing my source), or leave as-is?)***

\begin{figure}[h!]
\centering
\includegraphics[width=0.7\linewidth]{figures/gas-act-tab}
\caption{Helium Coolant Specific Activities \cite{noauthor_results_1990}}
\label{fig:gas-act-tab}
\end{figure}
\begin{figure}[h!]
\centering
\includegraphics[width=0.7\linewidth]{figures/dust-act-tab}
\caption{Pebble Dust Specific Activities \cite{noauthor_results_1990}}
\label{fig:dust-act-tab}
\end{figure}


\section{Modern HTGRs}

\subsection{PBMR}

The PBMR is a South African pebble bed HTGR design.  While it did not ultimately make it to construction, its design has offered invaluable insight to later HTGR pebble bed designs.  The PBMR is largely based on the German High Temperature Reactor (HTR) designs, and has a nameplate thermal power of 400 MW, with inlet-outlet temperatures of 500 \textdegree C to 900 \textdegree C.  It is a modular design, with each unit containing a graphite moderated, helium-cooled core housed in a steel pressure vessel.  In accident scenarios, the PBMR would rely on passive safety features using conduction and convection to provide cooling.

\begin{figure}[h!]
\centering
\includegraphics[width=0.5\linewidth]{figures/pbmr-v-cross-sect}
\caption{PBMR Schematic: Vertical Cross-section \cite{venter_pbmr_2005}}
\label{fig:pbmr-v-sect}
\end{figure}

Each core unit would hold around half a million pebbles, which used LEU based TRISO particles as the fuel form.  These TRISO particles are pressed into a 2.5cm radius graphite sphere, which then has an additional 0.5 cm thick layer of graphite pressed around it, to form a 3.0 cm radius pebble - around the size of a billiard ball.  The pebbles would undergo a six-pass multi-pass cycle to reach a target end burnup of 92,000 $\frac{MWd}{tU}$ \cite{venter_pbmr_2005}. 

\subsection{Next Generation Nuclear Plant (NGNP}

Like the PBMR, the NGNP did not make it to construction.  However, the work in analyzing reactor designs and materials is still applicable to other work.  The NGNP project downselected its design choices to two models - a prismatic HTGR and a pebble-bed HTGR.  While the NGNP project eventually opted for the Areva prismatic HTGR design \cite{noauthor_areva_nodate} due to reasons related to pebble costs, it was noted that, technologically speaking, there was no inherent advantage or disadvantage bewteen the two technologies \cite{inl_basis_2011}.

Even though the reactor didn't make it to construction or operation, the NGNP project, and the resulting white papers, have provided an invaluable licensing example for generation IV reactors with the NRC.  As much of the current NRC guidelines are based upon LWR technology, work towards validating HTGR materials still has to be developed \cite{lommers_ngnp_2012}.

\subsection{X-energy}

Based on experience working on the PBMR project, the X-energy Xe-100 is a 200 MWt HTGR pebble-bed SMR.  It is similar in design to all of its predecessors, featuring LEU TRISO particle fuel in 3.0 cm radius pebbles.  While the Xe-100, or similar demonstration plant, has not been built as of this publication, the project ***(company?  work?  design?)*** is still ongoing.  It is this reactor, and by extension, the PBMR, that the micro-reactor described in this thesis is most heavily influenced by.

The Xe-100 uses approximately 220,000 pebbles in a six-pass cycle, and fuel pebbles identical to the ones intended for the PBMR \cite{harlan_x-energy_2018}.  However, while the number of passes is the same, the target end burnup for the pebbles is higher, at 160,000 $\frac{MWd}{tU}$ \cite{agnihotri_intrinsically_2017}.  Another key difference from the PBMR beyond size is the lack of central reflector.

While the Xe-100 has not been built, there have been studies conducted by ORNL providing data on the production and material properties of the PBMR-type fuel pebble.


\chapter{Methodology}
\label{sec:methods}
\begin{table}[h!]
\centering
\caption{Geometric and Internal Core Parameters in the Sangamon Reactors}
\begin{tabular}{ c  c  c }
\hline
Parameter & Sangamon200 \cite{harlan_x-energy_2018}, \cite{harlan_ans_2017} & Sangamon20 \\
\hline
Thermal Power [MW] & 200 & 20 \\
Average Core Temperature [K] & 800 & 800 \\
Enrichment [wt\%] & 15.5\% & 19.75\% \\
Average Core Pressure [MPa] & 5.9 & 5.9 \\
Outer Core Radius [cm] & 216 & 165 \\
Outer Core Height [cm] & 1150 & 330 \\
Reflector Thickness [cm] & 92 & 75 \\
Number of Pebbles & 220,000 & 22,680 \\
Packing Fraction [\%] & 53.0 & 56.0 \\
\hline
\end{tabular}

\label{table:params1}
\end{table}

All neutronics simulations are performed using Serpent2.0 \cite{leppanenjaakko_serpent_2015} .  Pebbles are individually modeled, with locations generated using Serpent2.0's particle dispersal routine (***should I go into more detail on the dispersal routine?***).  Each pebble in the full-core model has the TRISO-filled "fueled-core" homogenized by volume.
\\
\begin{table}[h!]
\centering

\caption{Pebble Parameters}
\begin{tabular}{ c  c }
\hline
Parameter & Value \\
\hline
Fueled-Center Radius [cm] & 2.5 \\
Graphite Outer Shell Thickness [cm] & 0.5 \\
Total Radius [cm] & 3.0 \\
TRISO Particles per Pebble & 18,000 \\
\hline
\end{tabular}
\label{table:peb-params}
\end{table}

\begin{figure}[H]
\centering

\includegraphics[width=0.5\linewidth]{figures/pebble-zones.png}
\caption{Pebble Zones}
\label{fig:pebb-zone1}
\end{figure}

\begin{figure}[h!]
\centering
\includegraphics[width = 10cm]{figures/trisos-r-like-onions.png}
\caption{TRISO Particle Layers (not to scale)}
\label{fig:particle-layer}
\end{figure}
\begin{table}[h!]
\centering
\begin{tabular}{|| c || c |}
\hline
Parameter & Value \\
\hline \hline
Fueled-Center Radius [cm] & 2.5 \\
Graphite Outer Shell Thickness [cm] & 0.5cm \\
Total Radius [cm] & 3.0 \\
TRISO Particles per Pebble & 18,000 \\
\hline
\end{tabular}
\caption{Pebble Parameters}
\label{table:params2}
\end{table}
\begin{table}[h!]
\centering
\begin{tabular}{ c  c }
\hline
Parameter & Value \\
\hline 
Uranium Oxycarbide Kernel Radius [cm] & 0.02125 \\
Graphite Layer Thickness [cm] & 0.03075 \\
Inner Pyrolytic Carbon Layer Thickness [cm] & 0.03475 \\
Silicon Carbide Layer Thickness [cm] & 0.03825 \\
Outer Pyrolytic Carbon Layer Thickness [cm] & 0.04225 \\
\hline
\end{tabular}
\caption{Particle Parameters}
\label{table:params3}
\end{table}
\\
Fuel isotopic composition aside, the pebbles are identical in both reactor designs.  Both reactors feature a 6-month multi-pass cycle, with each pass through the core taking 6 months.  That is to say, a pebble will go through six 6-month passes before leaving the core.

\section{Sangamon0}
Sangamon0 is a 200 MWth helium cooled reactor.  It is an Xe-100 inspired design, and further informed by previous work on reactors such as the PBMR and NGNP.  Parameters are generally pulled from literature, or made by averaging given values in literature.  For reactor dimensions that are not specified, a rough estimate is approximated by assuming provided figures of a design are to scale, and converting measurements in pixels to cm.

Sangamon0 is still, however, a simplification of previously established designs.  The "cone" formed at the top and bottom of the reactor core is averaged to a flat surface, to create a cylindrical core shape.  The graphite reflector surrounds it, with no barriers between the reflector and helium/pebble-filled active core region.  In effect, the reflector is the "container" for the pebbles.  These are the only parts of the reactor that have been modeled.  It is assumed no control rods are being used.  In addition, the graphite reflector is modeled as solid.

While Sangamon0 is not the focus of this assessment, some neutronics features were determined to aid in Sangamon1's design.  A surface current detector was placed in the reflector, just inside the outer bound of the reflector, as shown in \ref{fig:det-place}.

\begin{figure}[h!]
\centering
\includegraphics{figures/detector-layout.png}
\caption{Detector Placement Inside Reflector}
\label{fig:det-place}
\end{figure}

This detector measures the outward neutron current (*** serpent outputs units of [number/s], is current still the best word? ***) in $[\frac{\#}{s}]$.  To arrive at the unit of $[\frac{\#}{cm^2s}]$ most are familiar with, the reported outward current is divided by the detector's surface area thusly:
\begin{equation}
J^+ [\frac{\#}{cm^2s}] = \frac{J^+ [\frac{\#}{s}]}{S_{det}[cm^2]}
\end{equation}

After accounting for the surface area, the outward current at the detector is 7.351e+11.

\section{Sangamon1}

Sangamon1 is a 20 MWth helium-cooled pebble bed reactor, fueled with 19.75\% enriched uranium oxycarbide.  While the capacity of Sangamon1 is 10\% that of Sangamon0, it is not possible to simply scale Sangamon0's dimensions down to 90\%.

\subsection{Inner Core Volume Determination}

The first assumption made in the scale-down is that Sangamon0 and Sangamon1 have the same power density, or $\frac{\text{kW}}{\text{g fuel}}$ (*** I called this "power density" as that is how serpent refers to this value.  But, given that it is per unit mass, is "specific power" a better term?***).

It is simple enough to calculate the mass of fuel in Sangamon0:

\begin{equation}
M_{fuel,0} = \frac{4}{3}\pi r_{uco}^3 \rho_{uco} n_{TRISO} n_{pebbles,0}
\end{equation}

Where
\begin{itemize}
\item $M_{fuel}$ is the mass of fuel in Sangamon0,
\item $r_{uco}$ is the radius of the UCO kernel inside a TRISO particle,
\item $\rho_{uco}$ is the density of UCO in $[\frac{g}{cc}]$
\item $n_{TRISO}$ is the number of TRISO particles in one pebble, and
\item $n_{pebbles}$ is the number of pebbles in Sangamon0.
\end{itemize}

Using the parameters in \ref{table:params1}, the power density of Sangamon0 and Sangamon1 is 0.11 $[\frac{kW}{g}]$.  With a power capacity of 20 MWth, one can calculate the total mass of UCO in Sangamon1 as
\begin{equation}
M_{fuel,1} = \frac{20*10^3 [kW]}{0.11[\frac{kW}{g}]} = 181818.18 [g]
\end{equation}
The mass of fuel in a single pebble can be found using the density of UCO and the total volume of UCO kernels in a single pebble, as above.  The number of pebbles in the entire reactor, then, is found by dividing the total mass of fuel by the mass of fuel in one pebble, as follows:

\begin{equation}
n_{pebbles,1} = \frac{M_{fuel,1}}{\frac{4}{3}r_{uco}^3n_{TRISO}\rho_{uco}}
\end{equation}

Rounding up, as it is not possible to have a partial pebble, we arrive at the value in \ref{table:params1}.

Knowing the number of pebbles is not enough - the exact dimensions of the active core region are still undefined.  To determine the volume of this space, the concept of the packing fraction - the ratio of the volume of objects (the pebbles) to the total volume of their container (the active core) - can be used.  The packing of even uniform objects in a 3-dimensional space is a complicated problem, often analyzed in the context of material studies or grain silos \cite{tulluri_analysis_nodate}.  For this reactor, it is assumed the pebble behavior can be described as random loose packing \cite{tulluri_analysis_nodate} - the pebbles have entered the core and fallen in an asystematic manner and have not been shaken to pack them in a more optimal fashion.  Such packing generally has a packing fraction in the range of 0.56 to 0.60 \cite{tulluri_analysis_nodate}.  Using the definition of the packing fraction, and previously defined terms, the active core volume is

\begin{equation}
V_{core,1} = \frac{ n_{pebbles,1}\frac{4}{3}\pi r_{pebble}^3 }{ \phi }
\end{equation}

Using the formula for the volume of a cylinder, one can plot possible values of $r_{core,1}$ and $h_{core,1}$ that satisfy the volume requirement.

\begin{figure}[H]
\centering
\includegraphics[width = 10cm]{figures/act-core-RH.png}
\caption{Curve of possible height and radii that satisfy the volume requirements imposed by packing fraction(s) $\eta_{pf}$}
\label{fig:rh-vol}
\end{figure}

It is also true that the most critical configurations for a cylinder are either a "fat" shape, where the height is equal to the diameter, or a flat "pancake" shape, that height is much less than diameter.  As a very flat shape is not very convienient for a reactor, the former is chosen.  The point indicated in \ref{fig:rh-vol} shows the radius and height selected for Sangamon1 - a radius of 90 cm, and a height of 180 cm.

\subsection{Graphite Reflector Thickness Determination}

The reflector must be sufficiently thick to keep the reactor critical, and protect the pressure vessel.  To ensure this, the outward current must be less than or equal to the outward current in Sangamon0 at the outer reflector boundary.  The detector layout in Sangamon1 is identical to \ref{fig:det-place}.

\section{Fuel Composition}

The isotopic composition of the pebbles is based on the number of passes the pebble has theoretically experienced.  There are seven possible pebble compositions, one for each of the six 6-month passes, plus an additional composition for fresh pebbles.  The seven pebble compositions are represented equally in number in the core, and the "kinds" of pebbles are randomly distributed throughout the core.

The exact isotopic composition is approximated by running a burnup calculation using Serpent2 for a single pebble in a unit square.  It uses a reflective boundary condition to simulate the presence of other pebbles or the reflector.  The void in the square is filled with helium.  While the full-core models homogenize the pebbles, the TRISO particles in the single pebble burnup models are individually modeled.  Just as with the location of the pebbles in the full core, the Serpent2 particle dispersal routine generated the TRISO particle locations.

\begin{figure}[h!]
\centering
\includegraphics[width = 10cm]{figures/burn-20.png}
\caption{Geometry of the Single-Pebble Burnup Calculation: Sangamon1}
\label{fig:burn-20}
\end{figure}

Once the isotopic compositions are determined, the pebbles are homogenized by volume, to improve performance.  The volume of a TRISO particle, and more specifically, a UCO kernel, is assumed constant.

\section{Reactor Sensitivity to Pebble Locations and Symmetry}

As the pebble locations and compositions are determined randomly, it is entirely possible to have "bands" in the reactor where multiple pebbles of same (or similar) burnup form lines or pockets.  In the interest of better characterizing the neutronics of the reactor, a sensitivity analysis tested different pebble compositon locations.  The "shuffling" test maintained the pebble locations, but changed what composition the individual pebbles were.  A second test analyzed the effects of utilizing a symmetry simplification, in order to improve computational speed.  The core was modeled using a $\frac{1}{6}$ slice.  For 6 tests, the area the representative slice was taken from changed as shown in \ref{fig:slicetest}.  In each test, all other parameters remain the same.

\begin{figure}[h!]
\centering
\includegraphics{figures/run-layout.png}
\caption{Symmetry Test Run Layouts}
\label{fig:slicetest}
\end{figure}

\chapter{Results}
results

general outline:

\begin{itemize}
\item full core model
	\begin{itemize}
	\item radial banding on outer edge due to pebble location
	\item to a lesser degree, axial banding on top/bottom
	\item while not symmetrical, 'hot spots' are fairly rare, and a general gradient in the fission rate (orange) can be seen.
	\item noticeable hotspots seem to be at the very top/bottom, in the middle, and at the edges in the radial image.  Of course, this isn't seen in the axial mesh, because the edges of the axial mesh are only integrating over a small amount of material, while in the radial mesh, all parts integrate over the entire height of the reactor.
	\item ***One point of interest (discuss in meeting?)*** is that theoretically, the centerline of the axial mesh should all be integrating over the same amount of material - a thickness corresponding to the diameter of the reactor, 165 inches, which is also the height.  However, the hotspots I see at the top and bottom in the center aren't carried through the centerline.  If the hotspots in the top and bottom of the axial mesh were simply because it integrated over more material, shouldn't the entire centerline be brighter?
	\item I do notice a *very slight* trend to the right side of the axial mesh, at least toward the bottom
	\item since the radial mesh is integrating over the same volume throughout, I expected it to show the sort of trend in fission rate I would expect from a cylindrical reactor - more intense at the center, tapering toward the edge.  Certainly, the reflector (blue) is doing that with scatters.  But why is the center so dark compared to the outside?  is the banding that intense?
		\begin{itemize}
		\item had a thought: the meshes integrate over space, right.  It so happens that the bands here are where pebbles are perfectly lining up.  What if this intensity at the edges is a misrepresentation, a limitation of how the mesh is made?  If integrating over space, and all the the space in that section are pebbles - ie, fuel, remember they are homogenized - then the effect is that the integrated space is more concentrated.  As you go towards the center, and pebble location becomes random, the same 'section' integrated in isn't just fuel pebbles anymore, there could also be coolant.  It's 'watering down' that section.  If you look at the axial mesh, at the left and right edges you can see exactly what one pebble looks like, and it's rather dark
		\end{itemize}
	\end{itemize}
	
\item sensitivity and symmetry
	\begin{itemize}
	\item no noticeable change in the k-eff between runs, for either switching pebble comps or going to a 1/6th core
	\item banding behavior highlights a potential issue in using symmetry, but for 1/6th did not significantly change the overall trends in the mesh figures
	\item notably, I did compare the meshes, and for the radial symmetry at least, is is quite literally just taking the corresponding portion of the full core control and reflecting it around (i expected a little bit of a difference at least)
	\end{itemize}
\item flux
	\begin{itemize}
	\item radial and axial fluxes, two group
	\item 2 group flux in a fresh and 6pass pebble
	\end{itemize}
	
\end{itemize}
\section{Full-Core Control Model}

\begin{figure}[H]
\centering
%
\begin{subfigure}{0.4\textwidth}
  \includegraphics[width=0.95\linewidth]{figures/burn-20-bstep0}
  \caption{Fresh}
  \label{fig:bstep0}
\end{subfigure}%
%
\begin{subfigure}{0.4\textwidth}
  \includegraphics[width=0.95\linewidth]{figures/burn-20-bstep1}
  \caption{Six Months}
  \label{fig:bstep1}
\end{subfigure}%

\caption{Serpent-generated mesh figures of the fission rate (hot color map) and thermal flux (cold color map) for the representative single-pebble at each depletion step.  A cold color map is from shades of whitish-blue (high) to blackish-blue (low) while the hot color map is from a whitish-yellow (high) to reddish-brown (low)}
\end{figure}

\begin{figure}[H]\ContinuedFloat
\centering

\begin{subfigure}{0.4\textwidth}
  \includegraphics[width=0.95\linewidth]{figures/burn-20-bstep2}
  \caption{Twelve Months}
  \label{fig:bstep2}
\end{subfigure}%
%
\begin{subfigure}{0.4\textwidth}
  \includegraphics[width=0.95\linewidth]{figures/burn-20-bstep3}
  \caption{Eighteen Months}
  \label{fig:bstep3}
\end{subfigure}%

\begin{subfigure}{0.4\textwidth}
  \includegraphics[width=0.95\linewidth]{figures/burn-20-bstep4}
  \caption{Twenty-Four Months}
  \label{fig:bstep4}
\end{subfigure}%
%
\begin{subfigure}{0.4\textwidth}
  \includegraphics[width=0.95\linewidth]{figures/burn-20-bstep5}
  \caption{Thirty Months}
  \label{fig:bstep5}
\end{subfigure}%

\begin{subfigure}{0.4\textwidth}
  \includegraphics[width=0.95\linewidth]{figures/burn-20-bstep6}
  \caption{Thirty-Six Months}
  \label{fig:bstep6}
\end{subfigure}%
%
\caption{Serpent-generated mesh figures of the fission rate (hot color map) and thermal flux (cold color map) for the representative single-pebble at each depletion step.  A cold color map is from shades of whitish-blue (high) to blackish-blue (low) while the hot color map is from a whitish-yellow (high) to reddish-brown (low). (cont)}
\label{fig:burn-meshes}
\end{figure}
\begin{itemize}
\item meshes for each burnup used in single pebble
\item pebbles are not homogenized
\item each pass is six months
\item pebble is 3.0 cm across, with helium to fill out the cube it is inscribed in
\item each corresponds to a fuel composition in the full core
\item can see the outside (helium, material: blue) become paler over time (fewer scatters?)
\item can see the pebble (sphere, mat: non-homogenized pebble) become darker over time (fewer fissions?)
\end{itemize}

\begin{figure}
\centering

\begin{subfigure}{0.45\textwidth}
  \includegraphics[width=0.95\linewidth]{figures/control/control-r}
  \caption{Radial Cross Section at y=0}
  \label{fig:bstep0}
\end{subfigure}%
%
\begin{subfigure}{0.45\textwidth}
  \includegraphics[width=0.95\linewidth]{figures/control/control-rm}
  \caption{Radial Mesh}
  \label{fig:bstep1}
\end{subfigure}

\begin{subfigure}{0.45\textwidth}
  \includegraphics[width=0.95\linewidth]{figures/control/control-v}
  \caption{Axial Cross Section at z=0 }
  \label{fig:bstep1}
\end{subfigure}
%
\begin{subfigure}{0.45\textwidth}
  \includegraphics[width=0.95\linewidth]{figures/control/control-vm}
  \caption{Axial Mesh}
  \label{fig:bstep1}
\end{subfigure}
%
\caption{Full Core}
\label{fig:control}
\end{figure}
\begin{itemize}
\item full core model
\item homogenized pebbles
\item no symmetry setting
\item rings due to pebbles naturally lining up along the outer edges due to the physical barrier of the graphite, then becoming more random towards the center
\item teal is helium
\item pink is graphite
\item pebbles green: lighter -> darker = fresh -> burnt
\item remember that the meshes are integrated to create the 2d representation, it is not the fission/scattering in that cross section alone
\end{itemize}

The burnup models of the single pebble inscribed in a cube of helium provide the middle-of-life compositions used in the Sangamon20 full-core models.  The exact compositions are provided in the appendix.  Overall, the behavior of the single-pebble burnup models over time are as expected.  As the pebble is more burned, the fission and scattering rates decrease in the fuel pebble and surrounding helium, respectively.  The presence of 'holes' in the mesh aren't unexpected - the packing fraction of TRISO particles in the pebbles is significantly lower than the packing fraction of pebbles in the reactor.

The homogenized pebbles in the full-core model are much more uniform throughout the pebble compared to the heterogeneous pebbles, as one might expect.  A more interesting feature is the presence of radial rings in the outer reactor core.  These rings are more clearly seen in the mesh, but as the geometry cross-section shows, this phenomenon is due to the physical layout of the pebbles.  Towards the outside, the pebbles have the walls of the reactor to line up against, and so their location is uniform.  As you approach the center, without a physical barrier, the pebbles are able to find more 'random' positions.  This phenomenon also makes sense when one considers the mechanics of the Serpent 2.0 growth and shake algorithm.  As discussed in the Methods chapter, the growth and shake factors are very small fractions of pebble radius.  The dispersal routine determines pebble locations iteratively, and once the initial center point is determined, moves and grows each pebble the minimum amount required to fit the routine's three criterion:  1) the particle is at its full size, 2) the particle doesn't overlap with any other particles,  and 3) the particle is fully contained in the boundaries of the container.  Naturally, then, any particle spawned relatively close to the outer bound would line up with it as the algorithm 'shakes' the particles.




\section{Sensitivity Tests}

\subsection{Effects of Symmetry Assumption}

\begin{figure}[H]
\centering

\begin{subfigure}{0.45\textwidth}
  \includegraphics[width=0.95\linewidth]{figures/60-120/60-120-r}
  \caption{Radial Cross Section at y=0}
  \label{fig:bstep0}
\end{subfigure}%
%
\begin{subfigure}{0.45\textwidth}
  \includegraphics[width=0.95\linewidth]{figures/60-120/60-120-rm}
  \caption{Radial Mesh}
  \label{fig:bstep1}
\end{subfigure}

\begin{subfigure}{0.45\textwidth}
  \includegraphics[width=0.95\linewidth]{figures/60-120/60-120-v}
  \caption{Axial Cross Section at z=0 }
  \label{fig:bstep1}
\end{subfigure}
%
\begin{subfigure}{0.45\textwidth}
  \includegraphics[width=0.95\linewidth]{figures/60-120/60-120-vm}
  \caption{Axial Mesh}
  \label{fig:bstep1}
\end{subfigure}
%
\caption{Sensitivity Analysis: $60^{\circ}$ - $120^{\circ}$}
\label{fig:60-120}
\end{figure}
\begin{itemize}
\item 1/6 core symmetry
\item uses the section of the core that is the 0-60 slice from the full core
\item periodic boundary condition between core 'slices' vacuum outside
\item the center cross section is slightly better to visualize the 'banding' or petal behavior of the pebbles, but the axial one is more representative of the banding patterns in the entire core.  Remember, the line the axial image cuts through to give us this image lays on the boundary of a slice (the line corresponding to the horizontal one in the radial geom cross section) this means that the patterns seen in the axial geom here are repeated at the borders of every slice.
\end{itemize}

\begin{figure}[h!]
\centering

\begin{subfigure}{0.45\textwidth}
  \includegraphics[width=0.95\linewidth]{figures/120-180/120-180-r}
  \caption{Radial Cross Section at y=0}
  \label{fig:bstep0}
\end{subfigure}%
%
\begin{subfigure}{0.45\textwidth}
  \includegraphics[width=0.95\linewidth]{figures/120-180/120-180-rm}
  \caption{Radial Mesh}
  \label{fig:bstep1}
\end{subfigure}

\begin{subfigure}{0.45\textwidth}
  \includegraphics[width=0.95\linewidth]{figures/120-180/120-180-v}
  \caption{Axial Cross Section at z=0 }
  \label{fig:bstep1}
\end{subfigure}
%
\begin{subfigure}{0.45\textwidth}
  \includegraphics[width=0.95\linewidth]{figures/120-180/120-180-vm}
  \caption{Axial Mesh}
  \label{fig:bstep1}
\end{subfigure}
%
\caption{Sensitivity Analysis: $120^{\circ}$ - $180^{\circ}$}
\label{fig:120-180}
\end{figure}
\begin{itemize}
\item as above, but the 120-180 degree slice
\end{itemize}

\begin{figure}[h!]
\centering

\begin{subfigure}{0.45\textwidth}
  \includegraphics[width=0.95\linewidth]{figures/180-240/180-240-r}
  \caption{Radial Cross Section at y=0}
  \label{fig:bstep0}
\end{subfigure}%
%
\begin{subfigure}{0.45\textwidth}
  \includegraphics[width=0.95\linewidth]{figures/180-240/180-240-rm}
  \caption{Radial Mesh}
  \label{fig:bstep1}
\end{subfigure}

\begin{subfigure}{0.45\textwidth}
  \includegraphics[width=0.95\linewidth]{figures/180-240/180-240-v}
  \caption{Axial Cross Section at z=0 }
  \label{fig:bstep1}
\end{subfigure}
%
\begin{subfigure}{0.45\textwidth}
  \includegraphics[width=0.95\linewidth]{figures/180-240/180-240-vm}
  \caption{Axial Mesh}
  \label{fig:bstep1}
\end{subfigure}
%
\caption{Sensitivity Analysis: $180^{\circ}$ - $240^{\circ}$}
\label{fig:180-240}
\end{figure}
\begin{itemize}
\item as above, but the 180-140 degree slice
\end{itemize}

\begin{figure}
\centering

\begin{subfigure}{0.45\textwidth}
  \includegraphics[width=0.95\linewidth]{figures/240-300/240-300-r}
  \caption{Radial Cross Section at y=0}
  \label{fig:bstep0}
\end{subfigure}%
%
\begin{subfigure}{0.45\textwidth}
  \includegraphics[width=0.95\linewidth]{figures/240-300/240-300-rm}
  \caption{Radial Mesh}
  \label{fig:bstep1}
\end{subfigure}

\begin{subfigure}{0.45\textwidth}
  \includegraphics[width=0.95\linewidth]{figures/240-300/240-300-v}
  \caption{Axial Cross Section at z=0 }
  \label{fig:bstep1}
\end{subfigure}
%
\begin{subfigure}{0.45\textwidth}
  \includegraphics[width=0.95\linewidth]{figures/240-300/240-300-vm}
  \caption{Axial Mesh}
  \label{fig:bstep1}
\end{subfigure}
%
\caption{Sensitivity Analysis: $240^{\circ}$ - $300^{\circ}$}
\label{fig:240-300}
\end{figure}
\begin{itemize}
\item as above, but the 240-300 degree slice
\end{itemize}

\begin{figure}[H]
\centering

\begin{subfigure}{0.45\textwidth}
  \includegraphics[width=0.95\linewidth]{figures/300-360/300-360-r}
  \caption{Radial Cross Section at y=0}
  \label{fig:bstep0}
\end{subfigure}%
%
\begin{subfigure}{0.45\textwidth}
  \includegraphics[width=0.95\linewidth]{figures/300-360/300-360-rm}
  \caption{Radial Mesh}
  \label{fig:bstep1}
\end{subfigure}

\begin{subfigure}{0.45\textwidth}
  \includegraphics[width=0.95\linewidth]{figures/300-360/300-360-v}
  \caption{Axial Cross Section at z=0 }
  \label{fig:bstep1}
\end{subfigure}
%
\begin{subfigure}{0.45\textwidth}
  \includegraphics[width=0.95\linewidth]{figures/300-360/300-360-vm}
  \caption{Axial Mesh}
  \label{fig:bstep1}
\end{subfigure}
%
\caption{Sensitivity Analysis: $300^{\circ}$ - $360^{\circ}$}
\label{fig:300-360}
\end{figure}
\begin{itemize}
\item as above, but the 300-360 degree slice
\end{itemize}

\subsection{Effects of Fuel Composition Shuffling}

\chapter{Conclusion}
This chapter provides a synopsis of the work we performed to characterize the Sangamon20 and its sensitivity to modeling simplifications.  The first section gives a general summary of the results from the heterogenization \autoref{res-hom}, symmetry (see \autoref{res-sym}) and shuffling (see \autoref{res-shuff}) and discussion.  The second, and final, section covers suggested future work

\section{Summary and Discussion}

Previous work in HTGR pebble-bed modeling noted that the specific lattice arrangement used had little effect \cite{turkmen_effect_2012}, \cite{karriem_mcnp_2001}, \cite{brown_stochastic_2005}.  The pebble-shuffling and symmetry tests support this observation in regards to a  random arrangement of the pebbles.

The symmetry test showed that for minimal banding --- areas of like pebbles creating streaks and rings once reflected --- the difference between a full core model and one using symmetry to simplify is between 0.63\% and 0.01\% in the outward neutron current with an average of 0.25\%, and between 0.15\% and 0.03\% for $k_{eff}$ with an average of 0.082\%.  Additionally, otherwise-identical models with a well-mixed core and different dispersal of pebbles differ between 0.33\% and 0.013\% in the outward current at the outer reflector edge with an average of 0.21\%, and between 0.11\% and 0.025\% in $k_{eff}$ with an average of 0.072\%.  

This is greater than the uncertainty of the either the outward current or $k_{eff}$, which are about 0.14\% and 0.05\% for the symmetry assumption, respectively, and 0.14\% and 0.05\% in the shuffling test, respectively, in the uppermost ranges of error, but less than or equal to the uncertainty in the lower ranges. For other pebble bed gas reactors, this suggests that one does not need to re-create the same reactor over and over with slightly different pebble arrangements in order to accurately characterize a core.  However, this may not remain true if the random pebble arrangement happens to cause similar pebble burnups to "lump" together, as shown by the upper ranges of error --- especially in the symmetry test --- being greater than error alone in some runs.

The heterogenization tests highlight the need for an accurate representation of TRISO particles.  While the overall differences between the fast and thermal flux profiles are within error, the homogenized pebbles will under-predict $k_{eff}$ by more than 4.0\%, while over-predicting the magnitude of the neutron energy lethargy-adjusted flux core spectrum by as much as 5.0\% at energies greater than $1 \times 10^{-04}$ [MeV].  The most dramatic changes to spectra are within the pebbles themselves, where high-energy neutron peaks are over-estimated in the fresh and sixth-pass pebbles by a factor of 2-4.  However, the total outward neutron current, used to gauge the effectiveness of the reflector, differed by only 0.349\%.  This is likely because the reflector is thick enough to minimize fast neutron leakage, which prevents the fast flux changes (see Figure \ref{fig:diff-fast}) in the active core from affecting the outer regions of the reflector. The thermal flux isn't changed to a degree unexplained by error (see Figure \ref{fig:diff-therm}), so the thermal neutron contribution to outward neutron current should be similar to that in the control as well.

For isotopic inventories, most isotopes either increased or decreased at a uniform rate with each pass through the core.  However, some isotopes, such as $^{239}Pu$ , reach a peak concentration in MOL, and subsequently decline.  When choosing a multi-pass cycle over an \acrfull{otto} fuel cycle, it is important to remember this behavior in the case of fuel failure that may require ejecting a pebble before its final pass.  This behavior also means that the pebble dust that accumulates in the reactor core and gas circulation loops will likely have a higher $^{239}Pu$ concentration than in spent fuel pebbles, as all pebble burnup states will contribute to the dust.


\section{Future Work}

The symmetry test showed that, with random mixing, simplifying the simulation by approximating the whole-core with a $\frac{1}{6}$ slice had minimal effects on the current and $k_{eff}$.  However, the 'banding' and petal-like pebble patterns this symmetry creates highlights a potential issue, however unlikely.  What if the random pebble dispersal happens to lump a number of the same or similar burnup pebbles together?  How would this affect a whole core model, or one using symmetry?  Future work could explore the effects of pebble 'lumping', such as the size of pebble-lump needed before it causes a noticeable effect, and what magnitude of effect can be expected.

Additionally, this design used an infinite lattice of like pebbles in the depletion model to arrive at an equilibrium composition.  It is possible to improve the accuracy of the equilibrium composition by using other methods.  By tracking compositions over time in the actual core, as opposed to an infinite lattice, or splitting the core into axial layers to track the pebble isotopic inventory as a function of the number of passes and current height in the core.  This would provide higher-fidelity, time-dependent data on pebble isotopics.  From there, an exploration of the resulting source term based on models of varying fidelity could provide insight on the accuracy of fuel composition data required to accurately determine source term and accident consequences.

The current design is slightly supercritical, and one could explore adding in an additional "half-pass" --- i.e., half of the sixth-pass pebbles go for a seventh pass, and the other half replaced with fresh pebbles.  This change would mean that there are half as many fresh pebbles in the core as there are currently, and the 7th pass pebbles, as they continue to burn in the core, would build up a higher concentration of neutron poisons and other fission products.  This would likely change the magnitude of the fast flux --- though it likely wouldn't change the overall shape --- and it would have an effect on the coolant and whole-core energy spectra, likely shifting it to lower energy ranges.  Alternatively, one could explore the addition of neutron-absorber pebbles to handle excess reactivity.

Finally, the pebble dispersal method used here does not account for gravity, which would make the pebbles settle closer together.  Without using a core that has a diameter which is an integer multiple of the pebble diameter, it is not possible to get a perfect close-pack arrangement (shaking a vessel will help achieve closer packing, but this is not possible here).  One could simulate the effects of gravity by dispersing the pebbles over a volume with a slightly shorter height, instead of the full height.  One could also split the core into axial layers, and apply a packing fraction equal to the theoretical maximum at the bottom-most layers, and reduce packing fraction as one moves up through the layers.  This would require careful tracking of the number of pebbles in each layer if the entire core uses a target number of fuel pebbles, but may provide a better estimate of pebble behavior in a system that does not otherwise incorporate gravity.  However, it is important to note that at a packing fraction of around 0.58, the model is already approaching the theoretical maximum packing fraction, so the difference this would make may be minimal.  At the same time, as pointed out in \cite{karriem_mcnp_2001}, even small changes to packing fraction and pebble arrangement may have a significant impact in small, low-leakage systems (such as a well-reflected \acrshort{smr} or microreactor), which is why exploring the effect the aforementioned changes may have could be important, even if the pebble arrangement doesn't appear to change much.

\backmatter
\nocite{*}
\bibliographystyle{IEEEtran}
\bibliography{bibliography}

\mainmatterWithoutReset
\appendix
\chapter{}

\label{app}
\section{Appendix A: Symmetry Test}
Appendix A contains the geometry cross sections, fission rate/thermal flux meshes, and image difference results from the other symmetry tests.


\begin{figure}[H]
\centering

\begin{subfigure}{0.45\textwidth}
  \includegraphics[width=0.95\linewidth]{figures/60-120/60-120-r}
  \caption{Radial Cross Section at y=0}
  \label{fig:bstep0}
\end{subfigure}%
%
\begin{subfigure}{0.45\textwidth}
  \includegraphics[width=0.95\linewidth]{figures/60-120/60-120-rm}
  \caption{Radial Mesh}
  \label{fig:bstep1}
\end{subfigure}

\begin{subfigure}{0.45\textwidth}
  \includegraphics[width=0.95\linewidth]{figures/60-120/60-120-v}
  \caption{Axial Cross Section at z=0 }
  \label{fig:bstep1}
\end{subfigure}
%
\begin{subfigure}{0.45\textwidth}
  \includegraphics[width=0.95\linewidth]{figures/60-120/60-120-vm}
  \caption{Axial Mesh}
  \label{fig:bstep1}
\end{subfigure}
%
\caption{Sensitivity Analysis: $60^{\circ}$ - $120^{\circ}$}
\label{fig:60-120}
\end{figure}
\begin{figure}[H]
\centering
\includegraphics[width=0.6\linewidth]{figures/60-120/diff-60-120}
\caption{An Image Generated by Subtracting \ref{fig:60-120-rm} from \ref{fig:controlb}.}
\label{fig:60-120-diff}
\end{figure}

Figure \ref{fig:60-120} provides fission rate and thermal flux visualization meshes for the symmetry test using the 60 - 120 degree slice.  Figure \ref{fig:60-120-diff} is the result of using image-difference between the control's full-core radial mesh and the symmetry test's mesh.

\begin{figure}[h!]
\centering

\begin{subfigure}{0.45\textwidth}
  \includegraphics[width=0.95\linewidth]{figures/120-180/120-180-r}
  \caption{Radial Cross Section at y=0}
  \label{fig:bstep0}
\end{subfigure}%
%
\begin{subfigure}{0.45\textwidth}
  \includegraphics[width=0.95\linewidth]{figures/120-180/120-180-rm}
  \caption{Radial Mesh}
  \label{fig:bstep1}
\end{subfigure}

\begin{subfigure}{0.45\textwidth}
  \includegraphics[width=0.95\linewidth]{figures/120-180/120-180-v}
  \caption{Axial Cross Section at z=0 }
  \label{fig:bstep1}
\end{subfigure}
%
\begin{subfigure}{0.45\textwidth}
  \includegraphics[width=0.95\linewidth]{figures/120-180/120-180-vm}
  \caption{Axial Mesh}
  \label{fig:bstep1}
\end{subfigure}
%
\caption{Sensitivity Analysis: $120^{\circ}$ - $180^{\circ}$}
\label{fig:120-180}
\end{figure}
\begin{figure}[H]
\centering
\includegraphics[width=0.6\linewidth]{figures/120-180/diff-120-180}
\caption{An Image Generated by Subtracting \ref{fig:120-180-rm} from \ref{fig:controlb}.}
\label{fig:120-180-diff}
\end{figure}

Figure \ref{fig:120-180} provides fission rate and thermal flux visualization meshes for the symmetry test using the 120 - 180 degree slice.  Figure \ref{fig:120-180-diff} is the result of using image-difference between the control's full-core radial mesh and the symmetry test's mesh.

\begin{figure}[h!]
\centering

\begin{subfigure}{0.45\textwidth}
  \includegraphics[width=0.95\linewidth]{figures/180-240/180-240-r}
  \caption{Radial Cross Section at y=0}
  \label{fig:bstep0}
\end{subfigure}%
%
\begin{subfigure}{0.45\textwidth}
  \includegraphics[width=0.95\linewidth]{figures/180-240/180-240-rm}
  \caption{Radial Mesh}
  \label{fig:bstep1}
\end{subfigure}

\begin{subfigure}{0.45\textwidth}
  \includegraphics[width=0.95\linewidth]{figures/180-240/180-240-v}
  \caption{Axial Cross Section at z=0 }
  \label{fig:bstep1}
\end{subfigure}
%
\begin{subfigure}{0.45\textwidth}
  \includegraphics[width=0.95\linewidth]{figures/180-240/180-240-vm}
  \caption{Axial Mesh}
  \label{fig:bstep1}
\end{subfigure}
%
\caption{Sensitivity Analysis: $180^{\circ}$ - $240^{\circ}$}
\label{fig:180-240}
\end{figure}
\begin{figure}[H]
\centering
\includegraphics[width=0.6\linewidth]{figures/180-240/diff-180-240}
\caption{An Image Generated by Subtracting \ref{fig:180-240-rm} from \ref{fig:controlb}.}
\label{fig:180-240-diff}
\end{figure}

Figure \ref{fig:180-240} provides fission rate and thermal flux visualization meshes for the symmetry test using the 180 - 240 degree slice.  Figure \ref{fig:180-240-diff} is the result of using image-difference between the control's full-core radial mesh and the symmetry test's mesh.

\begin{figure}
\centering

\begin{subfigure}{0.45\textwidth}
  \includegraphics[width=0.95\linewidth]{figures/240-300/240-300-r}
  \caption{Radial Cross Section at y=0}
  \label{fig:bstep0}
\end{subfigure}%
%
\begin{subfigure}{0.45\textwidth}
  \includegraphics[width=0.95\linewidth]{figures/240-300/240-300-rm}
  \caption{Radial Mesh}
  \label{fig:bstep1}
\end{subfigure}

\begin{subfigure}{0.45\textwidth}
  \includegraphics[width=0.95\linewidth]{figures/240-300/240-300-v}
  \caption{Axial Cross Section at z=0 }
  \label{fig:bstep1}
\end{subfigure}
%
\begin{subfigure}{0.45\textwidth}
  \includegraphics[width=0.95\linewidth]{figures/240-300/240-300-vm}
  \caption{Axial Mesh}
  \label{fig:bstep1}
\end{subfigure}
%
\caption{Sensitivity Analysis: $240^{\circ}$ - $300^{\circ}$}
\label{fig:240-300}
\end{figure}
\begin{figure}[H]
\centering
\includegraphics[width=0.6\linewidth]{figures/240-300/diff-240-300}
\caption{An Image Generated by Subtracting \ref{fig:240-300-rm} from \ref{fig:controlb}.}
\label{fig:240-300-diff}
\end{figure}

Figure \ref{fig:240-300} provides fission rate and thermal flux visualization meshes for the symmetry test using the 240 - 300 degree slice.  Figure \ref{fig:240-300-diff} is the result of using image-difference between the control's full-core radial mesh and the symmetry test's mesh.

\begin{figure}[H]
\centering

\begin{subfigure}{0.45\textwidth}
  \includegraphics[width=0.95\linewidth]{figures/300-360/300-360-r}
  \caption{Radial Cross Section at y=0}
  \label{fig:bstep0}
\end{subfigure}%
%
\begin{subfigure}{0.45\textwidth}
  \includegraphics[width=0.95\linewidth]{figures/300-360/300-360-rm}
  \caption{Radial Mesh}
  \label{fig:bstep1}
\end{subfigure}

\begin{subfigure}{0.45\textwidth}
  \includegraphics[width=0.95\linewidth]{figures/300-360/300-360-v}
  \caption{Axial Cross Section at z=0 }
  \label{fig:bstep1}
\end{subfigure}
%
\begin{subfigure}{0.45\textwidth}
  \includegraphics[width=0.95\linewidth]{figures/300-360/300-360-vm}
  \caption{Axial Mesh}
  \label{fig:bstep1}
\end{subfigure}
%
\caption{Sensitivity Analysis: $300^{\circ}$ - $360^{\circ}$}
\label{fig:300-360}
\end{figure}
\begin{figure}[H]
\centering
\includegraphics[width=0.6\linewidth]{figures/300-360/diff-300-360}
\caption{An Image Generated by Subtracting \ref{fig:300-360-rm} from \ref{fig:controlb}.}
\label{fig:300-360-diff}
\end{figure}

Figure \ref{fig:300-360} provides fission rate and thermal flux visualization meshes for the symmetry test using the 300 - 360 degree slice.  Figure \ref{fig:300-360-diff} is the result of using image-difference between the control's full-core radial mesh and the symmetry test's mesh.

To help explain the color difference plots, it may be helpful to go into further depth on RGB (red, green,blue) based colors and how the individual values correspond to colors when mixed.

In the RGB color format, values for red, green, and blue can range from 0 to 255.  The higher the value for a color, the more of it is present in the resulting color.  If the values for red, green, and blue are the same, then the resulting color is a shade of grey.  If all values are set to their maximum, 255, then the result is pure white.  If all values are 0, then the result is pure black.  In general, colors with low RGB values are darker than  ones with greater RGB values.

To understand why the differences in the active core are in green, it is useful to further describe basic color theory and complementary colors.  The fission rate meshes are shown in a hot color map, which ranges in color from an almost-white shade of yellow, to very dark browns.  In between these maximum and minimum shades are varying shades of yellow, orange, and brown.  To create a shade of yellow in RGB format, one uses a large amount of red and green.  To create the sort of almost-white yellow, one simply takes the base yellow, with large amounts of red and green, and increases the blue value (which, as described before, will transition the color to a lighter shade as all three RGB values approach the maximum of 255).  To move from yellow to an orangey-brown, one shifts the green value down.  Lowering red and green while keeping blue at a low value produces the darkest shades of brown seen in the color map.

\begin{figure}[H]
\centering
\includegraphics[width=0.6\linewidth]{figures/rgb-1}
\caption[An Example of RGB Color Values]{An example of RGB values.  If the value for red and blue are held constant, shifting the value for green up or down shifts the resulting color along the color gradient to the left of the green value, which ranges from red to yellow.  The arrows on the green gradient indicate what the current color is.  As one can see, moving the slider to the right - increasing the value of green - will make the color more yellow, while moving it to the left, or decreasing the green level, will shift it towards orange and red.}
\label{fig:rgb-1}
\end{figure}

Figure \ref{fig:rgb-1} gives an example of selecting a color using RGB values.  Image difference works by subtracting the RGB values from each other - for example, subtracting ( 200, 150, 50 ) from ( 100, 200, 75 ) results in ( 100, 50, 25 ).  Absolute values are used because negative values don't exist in RGB colors.  So, when  two colors which have contrasting values of green, and similar values of red and blue, the result is, of course, a shade of green.

In each of the image differences, the section used to approximate the entire core (for example, the section from 60 to 120 degrees in Figure \ref{fig:60-120-diff}).  Is very dark, which indicates that this region has little to no difference from the full-core control mesh.


\section{Appendix B: Shuffle Test}
Appendix B contains the geometry cross sections, fission rate/thermal flux meshes, and image difference results from the                                                                                                                                                                                                                                         shuffling tests.


\begin{figure}[H]
\centering

\begin{subfigure}{0.45\textwidth}
  \includegraphics[width=0.95\linewidth]{figures/1234560/1234560-r}
  \caption{Radial Cross Section at y=0}
  \label{fig:1234560-r}
\end{subfigure}%
%
\begin{subfigure}{0.45\textwidth}
  \includegraphics[width=0.95\linewidth]{figures/1234560/1234560-rm}
  \caption{Radial Mesh}
  \label{fig:1234560-rm}
\end{subfigure}

\begin{subfigure}{0.45\textwidth}
  \includegraphics[width=0.95\linewidth]{figures/1234560/1234560-v}
  \caption{Axial Cross Section at z=0 }
  \label{fig:1234560-v}
\end{subfigure}
%
\begin{subfigure}{0.45\textwidth}
  \includegraphics[width=0.95\linewidth]{figures/1234560/1234560-vm}
  \caption{Axial Mesh}
  \label{fig:1234560-vm}
\end{subfigure}
%
\caption{Shuffle Analysis: Run 1}
\label{fig:0-60}
\end{figure}
\begin{figure}[H]
\centering
\includegraphics[width=0.6\linewidth]{figures/shuffle/diff-1234560}
\caption{An Image Generated by Subtracting \ref{fig:1234560-rm} from \ref{fig:controlb}.}
\label{fig:diff-1234560}
\end{figure}

Figure \ref{fig:1234560} provides the thermal flux and fission rate meshes and geometric cross sections axially and radially.  Figure \ref{fig:diff-1234560} is the result of the image difference between the full core control mesh and Figure \ref{fig:1234560-rm}.

\begin{figure}[H]
\centering

\begin{subfigure}{0.45\textwidth}
  \includegraphics[width=0.95\linewidth]{figures/2345601/2345601-r}
  \caption{Radial Cross Section at y=0}
  \label{fig:2345601-r}
\end{subfigure}%
%
\begin{subfigure}{0.45\textwidth}
  \includegraphics[width=0.95\linewidth]{figures/2345601/2345601-rm}
  \caption{Radial Mesh}
  \label{fig:2345601-rm}
\end{subfigure}

\begin{subfigure}{0.45\textwidth}
  \includegraphics[width=0.95\linewidth]{figures/2345601/2345601-v}
  \caption{Axial Cross Section at z=0 }
  \label{fig:2345601-v}
\end{subfigure}
%
\begin{subfigure}{0.45\textwidth}
  \includegraphics[width=0.95\linewidth]{figures/2345601/2345601-vm}
  \caption{Axial Mesh}
  \label{fig:2345601-vm}
\end{subfigure}
%
\caption{Shuffle Analysis: Run 2}
\label{fig:0-60}
\end{figure}
\begin{figure}[H]
\centering
\includegraphics[width=0.6\linewidth]{figures/shuffle/diff-2345601}
\caption{An Image Generated by Subtracting Figure \ref{fig:2345601-rm} from Figure \ref{fig:controlb}.}
\label{fig:diff-2345601}
\end{figure}

Figure \ref{fig:2345601} provides the thermal flux and fission rate meshes and geometric cross sections axially and radially.  Figure \ref{fig:diff-2345601} is the result of the image difference between the full core control mesh and Figure \ref{fig:2345601-rm}.

\begin{figure}[H]
\centering

\begin{subfigure}{0.45\textwidth}
  \includegraphics[width=0.95\linewidth]{figures/3456012/3456012-r}
  \caption{Radial Cross Section at y=0}
  \label{fig:3456012-r}
\end{subfigure}%
%
\begin{subfigure}{0.45\textwidth}
  \includegraphics[width=0.95\linewidth]{figures/3456012/3456012-rm}
  \caption{Radial Mesh}
  \label{fig:3456012-rm}
\end{subfigure}

\begin{subfigure}{0.45\textwidth}
  \includegraphics[width=0.95\linewidth]{figures/3456012/3456012-v}
  \caption{Axial Cross Section at z=0 }
  \label{fig:3456012-v}
\end{subfigure}
%
\begin{subfigure}{0.45\textwidth}
  \includegraphics[width=0.95\linewidth]{figures/3456012/3456012-vm}
  \caption{Axial Mesh}
  \label{fig:3456012-vm}
\end{subfigure}
%
\caption{Shuffle Analysis: Run 3}
\label{fig:0-60}
\end{figure}
\begin{figure}[H]
\centering
\includegraphics[width=0.6\linewidth]{figures/shuffle/diff-3456012}
\caption{An Image Generated by Subtracting Figure \ref{fig:3456012-rm} from Figure \ref{fig:controlb}.}
\label{fig:diff-3456012}
\end{figure}

Figure \ref{fig:3456012} provides the thermal flux and fission rate meshes and geometric cross sections axially and radially.  Figure \ref{fig:diff-3456012} is the result of the image difference between the full core control mesh and Figure \ref{fig:3456012-rm}.

\begin{figure}[H]
\centering

\begin{subfigure}{0.45\textwidth}
  \includegraphics[width=0.95\linewidth]{figures/4560123/4560123-r}
  \caption{Radial Cross Section at y=0}
  \label{fig:4560123-r}
\end{subfigure}%
%
\begin{subfigure}{0.45\textwidth}
  \includegraphics[width=0.95\linewidth]{figures/4560123/4560123-rm}
  \caption{Radial Mesh}
  \label{fig:4560123-rm}
\end{subfigure}

\begin{subfigure}{0.45\textwidth}
  \includegraphics[width=0.95\linewidth]{figures/4560123/4560123-v}
  \caption{Axial Cross Section at z=0 }
  \label{fig:4560123-v}
\end{subfigure}
%
\begin{subfigure}{0.45\textwidth}
  \includegraphics[width=0.95\linewidth]{figures/4560123/4560123-vm}
  \caption{Axial Mesh}
  \label{fig:4560123-vm}
\end{subfigure}
%
\caption{Shuffle Analysis: Run 4}
\label{fig:4560123}
\end{figure}
\begin{figure}[H]
\centering
\includegraphics[width=0.6\linewidth]{figures/shuffle/diff-4560123}
\caption{An Image Generated by Subtracting \ref{fig:4560123-rm} from \ref{fig:controlb}.}
\label{fig:diff-4560123}
\end{figure}

Figure \ref{fig:4560123} provides the thermal flux and fission rate meshes and geometric cross sections axially and radially.  Figure \ref{fig:diff-4560123} is the result of the image difference between the full core control mesh and Figure \ref{fig:4560123-rm}.

\begin{figure}[H]
\centering

\begin{subfigure}{0.45\textwidth}
  \includegraphics[width=0.95\linewidth]{figures/5601234/5601234-r}
  \caption{Radial Cross Section at y=0}
  \label{fig:5601234-r}
\end{subfigure}%
%
\begin{subfigure}{0.45\textwidth}
  \includegraphics[width=0.95\linewidth]{figures/5601234/5601234-rm}
  \caption{Radial Mesh}
  \label{fig:5601234-rm}
\end{subfigure}

\begin{subfigure}{0.45\textwidth}
  \includegraphics[width=0.95\linewidth]{figures/5601234/5601234-v}
  \caption{Axial Cross Section at z=0 }
  \label{fig:5601234-v}
\end{subfigure}
%
\begin{subfigure}{0.45\textwidth}
  \includegraphics[width=0.95\linewidth]{figures/5601234/5601234-vm}
  \caption{Axial Mesh}
  \label{fig:5601234-vm}
\end{subfigure}
%
\caption{Shuffle Analysis: Run 5}
\label{fig:5601234}
\end{figure}
\begin{figure}[H]
\centering
\includegraphics[width=0.6\linewidth]{figures/shuffle/diff-5601234}
\caption{An Image Generated by Subtracting Figure \ref{fig:5601234-rm} from Figure \ref{fig:controlb}.}
\label{fig:diff-5601234}
\end{figure}

Figure \ref{fig:5601234} provides the thermal flux and fission rate meshes and geometric cross sections axially and radially.  Figure \ref{fig:diff-5601234} is the result of the image difference between the full core control mesh and Figure \ref{fig:5601234-rm}.

\begin{figure}[H]
\centering

\begin{subfigure}{0.45\textwidth}
  \includegraphics[width=0.95\linewidth]{figures/6012345/6012345-r}
  \caption{Radial Cross Section at y=0}
  \label{fig:6012345-r}
\end{subfigure}%
%
\begin{subfigure}{0.45\textwidth}
  \includegraphics[width=0.95\linewidth]{figures/6012345/6012345-rm}
  \caption{Radial Mesh}
  \label{fig:6012345-rm}
\end{subfigure}

\begin{subfigure}{0.45\textwidth}
  \includegraphics[width=0.95\linewidth]{figures/6012345/6012345-v}
  \caption{Axial Cross Section at z=0 }
  \label{fig:6012345-v}
\end{subfigure}
%
\begin{subfigure}{0.45\textwidth}
  \includegraphics[width=0.95\linewidth]{figures/6012345/6012345-vm}
  \caption{Axial Mesh}
  \label{fig:6012345-vm}
\end{subfigure}
%
\caption{Shuffle Analysis: Run 6}
\label{fig:0-60}
\end{figure}
\begin{figure}[H]
\centering
\includegraphics[width=0.6\linewidth]{figures/shuffle/diff-6012345}
\caption{An Image Generated by Subtracting Figure \ref{fig:6012345-rm} from Figure \ref{fig:controlb}.}
\label{fig:diff-6012345}
\end{figure}

Figure \ref{fig:6012345} provides the thermal flux and fission rate meshes and geometric cross sections axially and radially.  Figure \ref{fig:diff-6012345} is the result of the image difference between the full core control mesh and Figure \ref{fig:6012345-rm}.

Comparing the image difference results of Appendix B, the shuffling test, to Appendix A, the symmetry test, shows that there is a smaller difference caused by the shuffling tests overall.  The small differences in this particular test would most likely indicate that the core is well-mixed, i.e., that each bin in the vertical direction, along the z axis, has each of the 7 fuel compositions represented equally.  In theory, if certain regions were highlighted in bright green, it would indicate regions that are poorly mixed.

\chapter{}
\section{Shuffle Test}
\label{app-shuf}
Appendix B contains the geometry cross sections, fission rate/thermal flux meshes, and image difference results from the shuffling tests (see \autoref{meth-sens}, Table \ref{table:shuffle}), which were omitted from the main report for brevity.

Comparing the image difference results of Appendix B, the shuffling test, to Appendix A, the symmetry test, shows that the shuffling tests have a weaker effect on the fission rate and thermal flux than the symmetry tests.  The small differences in this particular test would most likely indicate that the core is generally well-mixed, i.e., that each bin in the vertical direction, along the z axis, has each of the 7 fuel compositions represented equally.


\begin{figure}[H]
\centering

\begin{subfigure}{0.45\textwidth}
  \includegraphics[width=0.95\linewidth]{figures/1234560/1234560-r}
  \caption{Radial Cross Section at y=0}
  \label{fig:1234560-r}
\end{subfigure}%
%
\begin{subfigure}{0.45\textwidth}
  \includegraphics[width=0.95\linewidth]{figures/1234560/1234560-rm}
  \caption{Radial Mesh}
  \label{fig:1234560-rm}
\end{subfigure}

\begin{subfigure}{0.45\textwidth}
  \includegraphics[width=0.95\linewidth]{figures/1234560/1234560-v}
  \caption{Axial Cross Section at z=0 }
  \label{fig:1234560-v}
\end{subfigure}
%
\begin{subfigure}{0.45\textwidth}
  \includegraphics[width=0.95\linewidth]{figures/1234560/1234560-vm}
  \caption{Axial Mesh}
  \label{fig:1234560-vm}
\end{subfigure}
%
\caption{Shuffle Analysis: Run 1}
\label{fig:0-60}
\end{figure}
\begin{figure}[H]
\centering
\includegraphics[width=0.6\linewidth]{figures/shuffle/diff-1234560}
\caption{An Image Generated by Subtracting \ref{fig:1234560-rm} from \ref{fig:controlb}.}
\label{fig:diff-1234560}
\end{figure}

Figure \ref{fig:1234560} provides the thermal flux and fission rate meshes and geometric cross sections axially and radially.  Figure \ref{fig:diff-1234560} is the result of the image difference between the full core control mesh and Figure \ref{fig:1234560-rm}.

\begin{figure}[H]
\centering

\begin{subfigure}{0.45\textwidth}
  \includegraphics[width=0.95\linewidth]{figures/2345601/2345601-r}
  \caption{Radial Cross Section at y=0}
  \label{fig:2345601-r}
\end{subfigure}%
%
\begin{subfigure}{0.45\textwidth}
  \includegraphics[width=0.95\linewidth]{figures/2345601/2345601-rm}
  \caption{Radial Mesh}
  \label{fig:2345601-rm}
\end{subfigure}

\begin{subfigure}{0.45\textwidth}
  \includegraphics[width=0.95\linewidth]{figures/2345601/2345601-v}
  \caption{Axial Cross Section at z=0 }
  \label{fig:2345601-v}
\end{subfigure}
%
\begin{subfigure}{0.45\textwidth}
  \includegraphics[width=0.95\linewidth]{figures/2345601/2345601-vm}
  \caption{Axial Mesh}
  \label{fig:2345601-vm}
\end{subfigure}
%
\caption{Shuffle Analysis: Run 2}
\label{fig:0-60}
\end{figure}
\begin{figure}[H]
\centering
\includegraphics[width=0.6\linewidth]{figures/shuffle/diff-2345601}
\caption{An Image Generated by Subtracting Figure \ref{fig:2345601-rm} from Figure \ref{fig:controlb}.}
\label{fig:diff-2345601}
\end{figure}

Figure \ref{fig:2345601} provides the thermal flux and fission rate meshes and geometric cross sections axially and radially.  Figure \ref{fig:diff-2345601} is the result of the image difference between the full core control mesh and Figure \ref{fig:2345601-rm}.

\begin{figure}[H]
\centering

\begin{subfigure}{0.45\textwidth}
  \includegraphics[width=0.95\linewidth]{figures/3456012/3456012-r}
  \caption{Radial Cross Section at y=0}
  \label{fig:3456012-r}
\end{subfigure}%
%
\begin{subfigure}{0.45\textwidth}
  \includegraphics[width=0.95\linewidth]{figures/3456012/3456012-rm}
  \caption{Radial Mesh}
  \label{fig:3456012-rm}
\end{subfigure}

\begin{subfigure}{0.45\textwidth}
  \includegraphics[width=0.95\linewidth]{figures/3456012/3456012-v}
  \caption{Axial Cross Section at z=0 }
  \label{fig:3456012-v}
\end{subfigure}
%
\begin{subfigure}{0.45\textwidth}
  \includegraphics[width=0.95\linewidth]{figures/3456012/3456012-vm}
  \caption{Axial Mesh}
  \label{fig:3456012-vm}
\end{subfigure}
%
\caption{Shuffle Analysis: Run 3}
\label{fig:0-60}
\end{figure}
\begin{figure}[H]
\centering
\includegraphics[width=0.6\linewidth]{figures/shuffle/diff-3456012}
\caption{An Image Generated by Subtracting Figure \ref{fig:3456012-rm} from Figure \ref{fig:controlb}.}
\label{fig:diff-3456012}
\end{figure}

Figure \ref{fig:3456012} provides the thermal flux and fission rate meshes and geometric cross sections axially and radially.  Figure \ref{fig:diff-3456012} is the result of the image difference between the full core control mesh and Figure \ref{fig:3456012-rm}.

\begin{figure}[H]
\centering

\begin{subfigure}{0.45\textwidth}
  \includegraphics[width=0.95\linewidth]{figures/4560123/4560123-r}
  \caption{Radial Cross Section at y=0}
  \label{fig:4560123-r}
\end{subfigure}%
%
\begin{subfigure}{0.45\textwidth}
  \includegraphics[width=0.95\linewidth]{figures/4560123/4560123-rm}
  \caption{Radial Mesh}
  \label{fig:4560123-rm}
\end{subfigure}

\begin{subfigure}{0.45\textwidth}
  \includegraphics[width=0.95\linewidth]{figures/4560123/4560123-v}
  \caption{Axial Cross Section at z=0 }
  \label{fig:4560123-v}
\end{subfigure}
%
\begin{subfigure}{0.45\textwidth}
  \includegraphics[width=0.95\linewidth]{figures/4560123/4560123-vm}
  \caption{Axial Mesh}
  \label{fig:4560123-vm}
\end{subfigure}
%
\caption{Shuffle Analysis: Run 4}
\label{fig:4560123}
\end{figure}
\begin{figure}[H]
\centering
\includegraphics[width=0.6\linewidth]{figures/shuffle/diff-4560123}
\caption{An Image Generated by Subtracting \ref{fig:4560123-rm} from \ref{fig:controlb}.}
\label{fig:diff-4560123}
\end{figure}

Figure \ref{fig:4560123} provides the thermal flux and fission rate meshes and geometric cross sections axially and radially.  Figure \ref{fig:diff-4560123} is the result of the image difference between the full core control mesh and Figure \ref{fig:4560123-rm}.

\begin{figure}[H]
\centering

\begin{subfigure}{0.45\textwidth}
  \includegraphics[width=0.95\linewidth]{figures/5601234/5601234-r}
  \caption{Radial Cross Section at y=0}
  \label{fig:5601234-r}
\end{subfigure}%
%
\begin{subfigure}{0.45\textwidth}
  \includegraphics[width=0.95\linewidth]{figures/5601234/5601234-rm}
  \caption{Radial Mesh}
  \label{fig:5601234-rm}
\end{subfigure}

\begin{subfigure}{0.45\textwidth}
  \includegraphics[width=0.95\linewidth]{figures/5601234/5601234-v}
  \caption{Axial Cross Section at z=0 }
  \label{fig:5601234-v}
\end{subfigure}
%
\begin{subfigure}{0.45\textwidth}
  \includegraphics[width=0.95\linewidth]{figures/5601234/5601234-vm}
  \caption{Axial Mesh}
  \label{fig:5601234-vm}
\end{subfigure}
%
\caption{Shuffle Analysis: Run 5}
\label{fig:5601234}
\end{figure}
\begin{figure}[H]
\centering
\includegraphics[width=0.6\linewidth]{figures/shuffle/diff-5601234}
\caption{An Image Generated by Subtracting Figure \ref{fig:5601234-rm} from Figure \ref{fig:controlb}.}
\label{fig:diff-5601234}
\end{figure}

Figure \ref{fig:5601234} provides the thermal flux and fission rate meshes and geometric cross sections axially and radially.  Figure \ref{fig:diff-5601234} is the result of the image difference between the full core control mesh and Figure \ref{fig:5601234-rm}.

\begin{figure}[H]
\centering

\begin{subfigure}{0.45\textwidth}
  \includegraphics[width=0.95\linewidth]{figures/6012345/6012345-r}
  \caption{Radial Cross Section at y=0}
  \label{fig:6012345-r}
\end{subfigure}%
%
\begin{subfigure}{0.45\textwidth}
  \includegraphics[width=0.95\linewidth]{figures/6012345/6012345-rm}
  \caption{Radial Mesh}
  \label{fig:6012345-rm}
\end{subfigure}

\begin{subfigure}{0.45\textwidth}
  \includegraphics[width=0.95\linewidth]{figures/6012345/6012345-v}
  \caption{Axial Cross Section at z=0 }
  \label{fig:6012345-v}
\end{subfigure}
%
\begin{subfigure}{0.45\textwidth}
  \includegraphics[width=0.95\linewidth]{figures/6012345/6012345-vm}
  \caption{Axial Mesh}
  \label{fig:6012345-vm}
\end{subfigure}
%
\caption{Shuffle Analysis: Run 6}
\label{fig:0-60}
\end{figure}
\begin{figure}[H]
\centering
\includegraphics[width=0.6\linewidth]{figures/shuffle/diff-6012345}
\caption{An Image Generated by Subtracting Figure \ref{fig:6012345-rm} from Figure \ref{fig:controlb}.}
\label{fig:diff-6012345}
\end{figure}

Figure \ref{fig:6012345} provides the thermal flux and fission rate meshes and geometric cross sections axially and radially.  Figure \ref{fig:diff-6012345} is the result of the image difference between the full core control mesh and Figure \ref{fig:6012345-rm}.

There are a few hotspots where small regions show a brighter patch of green.  The best example of this is in the fourth quadrant of Figure \ref{fig:diff-6012345}, near the $270^{\circ}$ line.  Hotspots such as this could be caused by poor mixing, which would make some pebbles occur in higher concentrations (and then cause a larger difference in the shuffle test, when the same region is now dominated by a different burnup) or by the shuffling putting a different pebble burnup in a region of lower or higher flux.  To investigate the source of this, the pebbles in a ~10 cm square column (10 cm in x and y, the height of the reactor in z) around the hotspot were found.  Table \ref{table:10cmpebb} shows the count of each pebble burnup.


\begin{table}[H]
\centering
\caption{Representation of Pebbles by Number of Passes in a 10 cm Square Rectangular Prism Surrounding an Image Difference Hotspot at Approximately x = 11 cm, y = -56 cm}
 \begin{tabularx}{0.35\textwidth}{c  c}
 	\hline
 	Pebble Pass & Number of Pebbles \\
 	\hline
 	Fresh & 20 \\
 	First Pass & 12 \\
 	Second Pass & 13 \\
 	Third Pass & 12 \\
 	Fourth Pass & 10 \\
 	Fifth Pass & 14 \\
 	Sixth Pass & 11 \\
 	\hline
 \end{tabularx}
\label{table:10cmpebb}
\end{table}

When this region was narrowed further, to a ~5 cm region, the poor mixing was more dramatic, as seen in Table \ref{table:5cmpebb}.


\begin{table}[H]
\centering
\caption{Representation of Pebbles by Number of Passes in a 5 cm Square Rectangular Prism Surrounding an Image Difference Hotspot at Approximately x = 11 cm, y = -56 cm}
 \begin{tabularx}{0.35\textwidth}{c  c}
 	\hline
 	Pebble Pass & Number of Pebbles \\
 	\hline
 	Fresh & 10\\
 	First Pass & 3 \\
 	Second Pass & 4 \\
 	Third Pass & 1 \\
 	Fourth Pass & 2 \\
 	Fifth Pass & 2 \\
 	Sixth Pass & 2 \\
 	\hline

 \end{tabularx}
\label{table:5cmpebb}
\end{table}

In the 10 cm square column, fresh pebbles originally make up 21.7\% of all pebbles in the region.  In the 5cm square column that is tighter around the hotspot, fresh pebbles make up 41.7\% of all pebbles.  The hotspot pointed out in Figure \ref{fig:diff-6012345} exists in some degree in Figures \ref{fig:diff-1234560}, \ref{fig:diff-2345601}, \ref{fig:diff-3456012}, \ref{fig:diff-4560123}, and \ref{fig:diff-5601234} --- it is simply brightest in Figure \ref{fig:diff-6012345} because the shuffle scheme corresponding to this Figure \ref{fig:diff-6012345} replaces fresh pebbles with 6-pass pebbles, which have the greatest disparity in burnup.


\end{document}
\endinput
%%
%% End of file `thesis-ex.tex'.
