\begin{figure}[h!]
\centering
%
\begin{subfigure}{0.25\textwidth}
  \includegraphics[width=0.95\linewidth]{figures/compositions/iodine}
  \caption{Iodine}
  \label{fig:i}
\end{subfigure}%
%
\begin{subfigure}{0.25\textwidth}
  \includegraphics[width=0.95\linewidth]{figures/compositions/xenon}
  \caption{Xenon}
  \label{fig:xe}
\end{subfigure}%
%
\begin{subfigure}{0.25\textwidth}
  \includegraphics[width=0.95\linewidth]{figures/compositions/caesium}
  \caption{Caesium}
  \label{fig:cs}
\end{subfigure}%

\begin{subfigure}{0.25\textwidth}
  \includegraphics[width=0.95\linewidth]{figures/compositions/radium}
  \caption{Radium}
  \label{fig:ra}
\end{subfigure}%
%
\begin{subfigure}{0.25\textwidth}
  \includegraphics[width=0.95\linewidth]{figures/compositions/thorium}
  \caption{thorium}
  \label{fig:th}
\end{subfigure}%
%
\begin{subfigure}{0.25\textwidth}
  \includegraphics[width=0.95\linewidth]{figures/compositions/uranium}
  \caption{Uranium}
  \label{fig:u}
\end{subfigure}%

%
\begin{subfigure}{0.25\textwidth}
  \includegraphics[width=0.95\linewidth]{figures/compositions/plutonium}
  \caption{Plutonium}
  \label{fig:pu}
\end{subfigure}%

\caption{Evolution of Certain Isotopic Concentrations in Pebbles over Six Six-Month Passes}
\label{fig:comps}
\end{figure}