A synopsis of results and discussion are provided in this chapter, alongside recommendations of future work.

\section{Summary and Discussion}

Previous work in HTGR pebble-bed modeling noted that the specific lattice arrangement used did not have a statistically significant effect \cite{turkmen_effect_2012}, \cite{karriem_mcnp_2001}, \cite{brown_stochastic_2005}.  The pebble-shuffling and symmetry tests support this observation in regards to a  random arrangement of the pebbles.

The symmetry test showed that for minimal banding - areas of like pebbles creating streaks and rings once reflected - the difference between a full core model and one using symmetry to simplify is less than 0.65\% in the outward neutron current, and less than 0.15\% for $k_{eff}$.  Additionally otherwise-identical models with a well-mixed core and different dispersal of pebbles differ less than 1.2\% in the outward current at the outer reflector edge, and less than 0.32\% in $k_{eff}$.  For reactor models beyond Sangamon20, this suggests that one does not need to re-create the same reactor over and over with slightly different pebble arrangements in order to accurately characterize a core.

The heterogenized tests highlight the need for an accurate representation of TRISO particles.  While the overall differences between the fast and thermal flux profiles are minimal, the homogenized pebbles will under-predict k-eff by more than 4.0\%, while over-predicting the magnitude of the neutron energy lethargy-adjusted flux core spectrum by as much as 5.0\%.  The most dramatic changes to spectra are within the pebbles themselves, where high-energy neutron peaks are over-estimated in the fresh and sixth-pass pebbles by a factor of 2-4.  Additionally, the total outward neutron current, which was used to gauge the effectiveness of the reflector, differed by only 0.349\%.  This is likely because the reflector is thick enough to minimize fast neutron leakage, which prevents the fast flux changes (see Figure \ref{fig:diff-fast} in the active core from affecting the outer regions of the reflector. The thermal flux isn't changed to a degree that could not be explained by error (see Figure \ref{fig:diff-therm}, so the thermal neutron current is also similar.

For isotopic inventories, most isotopes either increased or decreased at a uniform rate with each pass through the core.  However, some isotopes, such as $^{239}Pu$ , reach a peak concentration in MOL, and subsequently decline.  When choosing a once-through-then-out (OTTO) or mult-pass fuel cycle, one must decide which isotopes it is most important to minimize.


\section{Future Work}

The symmetry test showed that, with random mixing, simplifying the model by approximating the whole-core with only a slice of it had minimal effects for a one-sixth symmetry.  However, the 'banding' and petal-like pebble patterns this symmetry creates highlights a potential issue, however unlikely.  What if the random pebble dispersal happens to lump a large number of the same or similar burnup pebbles together?  How would this affect a whole core model, or a model using symmetry?  Future work could explore the effects of pebble 'lumping', such as the size of pebble-lump needed before an effect is seen in the core model.

Additionally, this reactor model used an infinite lattice of like pebbles in the depletion model to arrive at an equilibrium composition.  It is possible to improve the accuracy of the equilibrium composition.  By tracking compositions over time in the actual core, as opposed to an infinite lattice, or splitting the core into axial layers to track the pebble isotopic inventory as a function of the number of passes and current height in the core.

The current model is not thermodynamically optimal, and future work could adjust the height-diameter ratio, provided it follows \ref{fig:rh-vol}.  Given that there is a slight excess reactivity, there should be room to shift to a slightly less critical shape that is more thermally beneficial.  If there is still a slight excess reactivity at this point, one could explore adding in an additional "half-pass" - i.e., half of the sixth-pass pebbles go for a seventh pass, and the other half are removed and replaced with fresh pebbles.  Alternatively, one could explore the addition of neutron-absorber pebbles to handle excess reactivity.

Finally, the pebble dispersal method used here does not account for gravity, which would make the pebbles settle closer together.  Without using a core that has a diameter which is an integer multiple of the pebble diameter, it is not possible to get a perfect close-pack arrangement (generally speaking, shaking a vessel will help achieve closer packing, but this is not possible here).  One could simulate the effects of gravity by dispersing the pebbles not over the whole volume, but rather a volume with a slightly shorter height.  However, it is important to note that at a packing fraction of around 0.58, the model is already approaching the theoretical maximum packing fraction, so the difference this would make may be minimal.