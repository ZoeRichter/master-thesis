\label{conc}

This chapter provides a synopsis of the work we performed to characterize the Sangamon20 and its sensitivity to modeling simplifications.  The first section gives a general summary of the results from the heterogenization (see \autoref{res-hom}), symmetry (see \autoref{res-sym}) and shuffling (see \autoref{res-shuff}) tests, and discusses them.  The second section covers suggested future work

\section{Summary and Discussion}

Previous work in HTGR pebble-bed modeling noted that the specific lattice arrangement used had little effect \cite{turkmen_effect_2012}, \cite{karriem_mcnp_2001}, \cite{brown_stochastic_2005}.  The pebble-shuffling and symmetry tests support this observation in regards to a  random arrangement of the pebbles.  Table \ref{table:con-tab1} below gives a summary of the range of errors, average error, and the associated control parameter uncertainty for the outward neutron current and $k_{eff}$ values for the symmetry and shuffling tests.

\begin{table}[h!]
\centering
\caption{Symmetry and Shuffle Test Error Ranges, Average Values, and Uncertainty in the Associated Control Model Value}
\begin{tabular}{ c  c  c  c }
\hline
Parameter & Error Range & Error Average & Associated Control Uncertainty \\
\hline
Outward Current: Symmetry Test & 0.682\% - 0.046\% & 0.145\% & 0.124\% \\
$k_{eff}$: Symmetry Test & 0.192\% - 0.074\% & 0.0789\% & $\pm$ 0.0545\% \\
Outward Current: Shuffle Test & 0.515\% - 0.032\% & 0.229\% & 0.124\% \\
$k_{eff}$: Shuffle Test & 0.110\% - 0.010\% & 0.05\% & $\pm$ 0.0545\% \\
\hline
\end{tabular}

\label{table:con-tab1}
\end{table}

Overall, the error values in the lower ranges, and the averages, tend to be very near, or less than  the uncertainty in the control model.  However, the outward current error upper ranges --- especially in the symmetry test --- can be much greater than the "base" model's uncertainty, by a factor of around 5.5 in the symmetry test.  On the whole, the error in $k_{eff}$ is much less than the error in the outward current, and the shuffling test had lower minimum and maximum error values than the symmetry test --- but averaged higher.  However, the differences in both were less than the differences observed from heterogenization versus homogenization

The heterogenization tests highlight the need for an accurate representation of TRISO particles.  While the overall differences between the fast and thermal flux profiles are within error, the homogenized pebbles will under-predict $k_{eff}$ by more than 4.0\%, while over-predicting the magnitude of the neutron energy lethargy-adjusted flux core spectrum by as much as 5.0\% at energies greater than $1 \times 10^{-04}$ [MeV].  The most dramatic changes to spectra are within the pebbles themselves, where high-energy neutron peaks are over-estimated in the fresh and sixth-pass pebbles by a factor of 2-4 due to the effects of self-shielding.  However, the total outward neutron current, used to gauge the effectiveness of the reflector, differed by only 0.349\%.  This is likely because the reflector is thick enough to minimize fast neutron leakage, which prevents the fast flux changes (see Figure \ref{fig:diff-fast}) in the active core from affecting the outer regions of the reflector. The thermal flux isn't changed to a degree unexplained by error (see Figure \ref{fig:diff-therm}), so the thermal neutron contribution to outward neutron current should be similar to that in the control as well.

For isotopic inventories, most isotopes either increased or decreased at a uniform rate with each pass through the core.  However, some isotopes, such as $^{239}Pu$ , reach a peak concentration in MOL, and subsequently decline.  When choosing a multi-pass cycle over an \acrfull{otto} fuel cycle, it is important to remember this behavior in the case of fuel failure that may require ejecting a pebble before its final pass.  This behavior also means that the pebble dust that accumulates in the reactor core and gas circulation loops will likely have a higher $^{239}Pu$ concentration than in spent fuel pebbles, as all pebble burnup states will contribute to the dust.


\section{Future Work}

The symmetry test showed that, with random mixing, simplifying the simulation by approximating the whole-core with a $\frac{1}{6}$ slice had minimal effects on the current and $k_{eff}$.  However, the 'banding' and petal-like pebble patterns this symmetry creates highlights a potential issue, however unlikely.  What if the random pebble dispersal happens to lump a number of the same or similar burnup pebbles together?  How would this affect a whole core model, or one using symmetry?  Future work could explore the effects of pebble 'lumping', such as the size of pebble-lump needed before it causes a noticeable effect, and what magnitude of effect can be expected.

Additionally, this design used an infinite lattice of like pebbles in the depletion model to arrive at an equilibrium composition.  It is possible to improve the accuracy of the equilibrium composition by using other methods.  By tracking compositions over time in the actual core, as opposed to an infinite lattice, or splitting the core into axial layers to track the pebble isotopic inventory as a function of the number of passes and current height in the core.  This would provide higher-fidelity, time-dependent data on pebble isotopics.  From there, an exploration of the resulting source term based on models of varying fidelity could provide insight on the accuracy of fuel composition data required to accurately determine source term and accident consequences.

The current design is slightly supercritical, and one could explore adding in an additional "half-pass" --- i.e., half of the sixth-pass pebbles go for a seventh pass, and the other half replaced with fresh pebbles.  This change would mean that there are half as many fresh pebbles in the core as there are currently, and the 7th pass pebbles, as they continue to burn in the core, would build up a higher concentration of neutron poisons and other fission products.  This would likely change the magnitude of the fast flux --- though it likely wouldn't change the overall shape --- and it would have an effect on the coolant and whole-core energy spectra, likely shifting it to lower energy ranges.  Alternatively, one could explore the addition of neutron-absorber pebbles to handle excess reactivity.

In the pebble spectra, the small size of a single pebble detector leads to high uncertainties.  While the coolant spectrum --- which has a much lower uncertainty --- can provide a general idea of the averaged pebble spectra, it cannot provide fine detail.  A future investigation using a multi-pebble detector will be necessary before the pebble spectra should be used in other calculations or analysis.

Finally, the pebble dispersal method used here does not account for gravity, which would make the pebbles settle closer together.  Without using a core that has a diameter which is an integer multiple of the pebble diameter, it is not possible to get a perfect close-pack arrangement (shaking a vessel will help achieve closer packing, but this is not possible here).  One could simulate the effects of gravity by dispersing the pebbles over a volume with a slightly shorter height, instead of the full height.  One could also split the core into axial layers, and apply a packing fraction equal to the theoretical maximum at the bottom-most layers, and reduce packing fraction as one moves up through the layers.  This would require careful tracking of the number of pebbles in each layer if the entire core uses a target number of fuel pebbles, but may provide a better estimate of pebble behavior in a system that does not otherwise incorporate gravity.  However, it is important to note that at a packing fraction of around 0.58, the model is already approaching the theoretical maximum packing fraction, so the difference this would make may be minimal.  At the same time, as pointed out in \cite{karriem_mcnp_2001}, even small changes to packing fraction and pebble arrangement may have a significant impact in small, low-leakage systems (such as a well-reflected \acrshort{smr} or microreactor), which is why exploring the effect the aforementioned changes may have could be important, even if the pebble arrangement doesn't appear to change much.