\section{Summary and Discussion}

Previous work in HTGR pebble-bed modeling noticed that the specific lattice arrangement used, by-and-large, did not significantly affect results.  The pebble-shuffling test combined with the symmetry test support this observation in regards to a completely random arrangement of pebbles.

The symmetry test showed that for minimal banding - areas of same pebbles creating streaks and rings once reflected - the difference between a full core model and one using symmetry to simplify is minimal.  Additionally, for models which assume a well-mixed core, the differences between otherwise-identical models with a different dispersal of pebbles are minimal.  For other models, this suggests that one does not need to re-create the same reactor over and over with slightly different pebble arrangements in order to accurately characterize a core.

The heterogenized tests highlight the need for an accurate representation of TRISO particles.  While the overall differences between the flux profiles are minimal, the homogenized pebbles will under-predict k-eff by more than 4.0\%, while over-predicting the magnitude of the neutron energy lethargy-adjusted flux spectra by as much as 5.0\%.  The most significant change is in the spectra within the pebbles themselves, where high-energy neutron peaks are over-estimated in the fresh and six-pass pebbles by a factor of 2-4.  Additionally, the outer current, which was used to gauge the effectiveness of the reflector, did not significantly change between the two models, likely because the reflector is unchanged between the two versions, and the reflector is thick enough to thermalize the fast neutrons entering from the core before they reach the outer edge, so changes in the fast neutron population in the active core don't have a noticeable effect on the outermost reaches.

For isotopic inventories, most isotopes either increased or decreased at a uniform rate with each pass through the core.  However, some isotopes, such as Pu-239, reach a peak concentration in MOL, and subsequently decline.  The isotopes that increase over pebble lifetime versus decrease is a point of consideration when choosing between multipass and OTTO fuel cycles.


\section{Future Work}

The symmetry test showed that, with random mixing, simplifying the model by approximating the whole-core with only a slice of it had minimal effects for a $\frac{1}{6}$ and greater symmetry.  However, the 'banding' and petal-like pebble patterns this symmetry created highlights a potential issue, however unlikely.  What if the random pebble dispersal happens to lump a large number of same or similar burnup pebbles together?  How would this affect a whole core model?  What of a model using symmetry?  Future work could explore the effects of pebble 'lumping', such as the size of pebble-lump needed before an effect is seen in the core model.

Additionally, this reactor model used an infinite lattice of like pebbles in the depletion model to arrive at an equilibrium composition.  While this is a fine first-guess, it is possible to improve the accuracy of the equilibrium composition.  For example, one could track compositions over time in the actual core, as opposed to an infinite lattice, or split the core into axial layers, and track the pebble isotopic inventory not simply as a function of the number of passes, but passes and current height in the core.

The current model is not thermodynamically optimal, and future work could adjust the height-diameter ratio, provided it follows \ref{fig-rh-vol.tex}.  Given that there is a slight excess reactivity, there should be room to shift to a slightly less critical shape that is more thermally beneficial.  If there is still a slight excess reactivity at this point, one could explore adding in an additional "half-pass" - i.e., half of the six-pass pebbles go for a seventh pass, and the other half are removed and replaced with fresh pebbles.  Alternatively, one could explore the addition of absorber pebbles to handle excess reactivity.

Finally, the pebble dispersal method used here does not account for gravity, which would make the pebbles settle a bit closer together.  Without shaking the core (not recommended, it would almost certainly make the issue of pebble dust worse, if not crack pebbles entirely) or using a core that has a diameter which is an integer multiple of the pebble diameter, it is not possible to get a perfect close-pack arrangement.  One could simulate the effects of gravity by dispersing the pebbles not over the whole volume, but rather a volume with a slightly shorter height.  However, it is important to note that at a packing fraction of around 0.58, the model is already approaching the theoretical maximum packing fraction, so the difference this would make may be minimal.